\chapter{Related Work}
\label{related}

% Notes
% HIPACC:
% * source - to - source C like compiler
% * generate CUDA, openCL, Renderscripts target GPU
% * make use of different memory hierarchy based on limit set of access pattern
% * support fix set of reduction function
% * unroll only if kernel size is known
% * heuristic pick best configuration (similar to our DSE)
%
% Rigel:
% * Generate verilog
% * Coarse-grain pipeline
% * very similar control semantics to plasticine. Use
% tokens/back pressure to allow each pipeline stage to fire at their perspective rate
% * Multi-rate line buffer template
%
% Darkroom
% * target ASIC, FPGA, fast CPU
% * generate structured verilog
% * line buffer
% * scheduling for inner loop pipeline. Use ILP to improve pipeline delay
%
% PolyMage
% * point=wise, stencil, sampling, operation
% * functional (in a kind of awkward way)
% * python
% * generate openMP/C++
% * generate high-level synthesis tool
% * line buffer

\section{Image Processing DSLs}
Recently proposed image processing DSLs provide high-level specifications for targeting various
accelerator platforms, including GPUs and FPGAs.%~\cite{hegarty2014darkroom, membarth2016hipa, hegarty2016rigel, chugh2016dsl}.
The narrow domain allows these DSLs to offer more concise abstractions for specifying stencil operations.
%Halide~\cite{pldi13halide}, for example, uses a declarative programming model which
%entirely hides the loop structure of stencils from the programmer.
When targeting accelerators, these languages usually rely on source-to-source translation.
$HIPA^{CC}$~\cite{membarth2016hipa}, for example, uses a
source-to-source compiler from a C-like front-end to generate CUDA, OpenCL, and Renderscript for
targeting GPUs.
Recent work on Halide~\cite{pldi13halide} has demonstrated targeting heterogeneous systems, including the Xilinx Zynq's FPGA and ARM cores, by generating intermediate C++ and Vivado HLS~\cite{halidefpga}.
Rigel~\cite{hegarty2016rigel} and Darkroom~\cite{hegarty2014darkroom} generate Verilog,
and PolyMage~\cite{chugh2016dsl} generates OpenMP and C++ for high-level synthesis.
Rigel and Darkroom support generation of specialized memory structures on FPGAs, such as line buffers, in order to capture
reuse. $HIPA^{CC}$ can infer memory hierarchy on GPUs from a fix set of access patterns.
These DSLs capture parallelism within a given stencil, typically across image channels
and across the image processing pipeline.

Compared to image processing DSLs, Spatial is more general and provides a
lower level of abstraction. Spatial can express pipelining and unrolling for arbitrary loop hierarchies and explicitly exposes the memory hierarchy while automatically banking, buffering, and duplicating structures for arbitrary
access patterns. These features, along with Spatial's design tuning
capabilities, make Spatial a natural fit as an optimizing backend target for image processing DSLs.

\section{Tiling}
Previous work on automated loop tiling has largely focused on tiling imperative programs
using polyhedral analysis~\cite{bondhugula08,pouchet10phd}.
There are many existing tools---such as Pluto~\cite{pluto08pldi},
PoCC~\cite{pouchet11popl}, CHiLL~\cite{chen2008chill},
and Polly~\cite{grosser2012polly}---that use polyhedral analysis
to automatically tile and parallelize programs.  These tools restrict memory
accesses within loops to affine functions of the loop iterators.
As a consequence, while they perform well on affine sections of programs,
they fail on commonly occurring data-dependent operations
like \emph{filters} and \emph{groupBys}~\cite{benabderrahmane10cc}. In order to handle these operations,
recent work has proposed using preprocessing steps which segment programs into affine
and non-affine sections prior to running polyhedral analysis tools~\cite{venkat}.

While the above work focused on the analysis of imperative programs, our work
analyzes functional parallel patterns, which offer a strictly higher-level representation
than simple imperative \emph{for} loops.
In this paper, we show that because of the additional semantic information
available in patterns like \emph{groupBy} and \emph{filter},
parallel patterns can be automatically tiled using
simple transformation rules, without the restriction that all memory accesses
are purely affine.
Little previous work has been done on automated tiling of functional
programs composed of arbitrarily nested parallel patterns.
Hielscher proposes a set of formal rules for tiling parallel operators \emph{map}, \emph{reduce}, and \emph{scan}
in the Parakeet JIT compiler, but these rules can be applied only for a small subset of nesting combinations \cite{parakeet}.
Spartan~\cite{spartan} is a runtime system with a set of high-level operators
(e.g., \emph{map} and \emph{reduce}) on multi-dimensional arrays, which
automatically tiles and distributes the arrays in a way that minimizes the
communication cost between nodes in cluster environments. In contrast to
our work, Spartan
operates on a tiled representation for distributed CPU computation and does attempt to optimize
performance on individual compute units.

%\begin{itemize}
%\item Tiling hyperplane method\cite{bondhugula08}
%\item Programming for Parallelism and Locality with Hierarchically Tiled Arrays \cite{bikshandi}
%\item Parameterized tiled loops for free\cite{renga07}
%\item Analytical bounds for optimal tile size selection\cite{shirako12}
%\item Tiling parallel patterns thesis
%\end{itemize}

\section{Hardware From High-level Languages}
The problem of generating hardware from high-level languages has been widely studied for
decades.  CHiMPS~\cite{chimps} generates hardware from ANSI C code by
mapping each C language construct in a data-flow graph to an HDL block.
Kiwi~\cite{kiwi} translates a set of C\# parallel constructs (e.g.,
\emph{event}, \emph{monitor}, and \emph{lock}) to corresponding hardware units.
Bluespec~\cite{bluespec} generates hardware from purely functional descriptions
based on Haskell.  Chisel~\cite{chisel} is an embedded language in Scala
for hardware generation.  AutoPilot~\cite{autopilot} is a commercial HLS
tool that generates hardware from C/C++/SystemC languages.  Despite their
success in raising the level of abstraction compared to hardware description
languages, programmers are still required to write programs at a low-level and
express how computations are pipelined and parallelized. Our work
abstracts away the implementation details from programmers by using high-level
parallel patterns, and applies compiler transformations to
automatically pipeline and parallelize operations and exploit on-chip memory
for locality.

Existing hardware synthesis tools are limited in their
ability to automatically infer and generate coarse-grained pipelines. A traditional
software pipelining approach is typically used on innermost loop bodies consisting
only of primitive operations. Optimizations like \emph{unroll-and-jam}, and \emph{unroll-and-squash}~\cite{unrollSquash}
also attempt to exploit pipelined parallelism, but target outer parallel loops
with inner sequential loops. To pipeline imperfectly nested loops, some
commercial high-level synthesis tools like Vivado~\cite{vivadohls} unroll all inner loops and then employ traditional
software pipelining. Not only does this approach generate needlessly large designs for large benchmarks,
it also suffers from long compilation times due to the complexity in scheduling a large number of unrolled instructions.
More recent works like ElasticFlow~\cite{elasticFlow} and CGPA~\cite{cgpa} generate coarse-grained
pipelines using FIFOs in between stages for communication. However, they handle only a restricted form of data access
patterns and a restricted number of nesting levels. Our metapipelining technique is more general than previous approaches because:
(i) metapipeline stages are decoupled using double buffers, therefore not restricting data access patterns, (ii) metapipelines
are easily composed and nested to any number of levels, and (iii) metapipelines can handle
dynamic rate mismatches as they use asynchronous handshaking for inter-stage synchronization, thereby obviating the need
to calculate static initiation interval as well as knowing loop trip counts ahead of time.


Recent work has explored using polyhedral analysis with HLS to optimize for data
locality on FPGAs~\cite{pouchet13fpga}.  Using polyhedral analysis, the compiler
is able to promote memory references to on-chip memory and parallelize
independent loop iterations with more hardware units.  However, the compiler is
not able to analyze loops that include non-affine accesses, limiting the
coverage of applications that can be generated for hardware. Our work can
handle parallel patterns with non-affine accesses by inferring required
hardware blocks (e.g., FIFOs and CAMs) for non-affine accesses, while
aggressively using on-chip memory for affine parts.

As high-level parallel patterns become increasingly popular to overcome the
shortcomings of C based languages, researchers have recently studied generating
hardware from functional parallel patterns.  Lime~\cite{auerbach10lime}
embeds high-level computational patterns (e.g., \emph{map}, \emph{reduce}, \emph{split}, and \emph{join}) in
Java and automatically targets CPUs, GPUs, and FPGAs without modifying the
code.  Our compiler manages a broader set of parallel patterns (e.g., \emph{groupBy})
and applies transformations even when patterns are nested,
which is common in a large number of real-world applications.  Recent work has
explored targeting nested parallel patterns to
FPGAs~\cite{george14fpl}. By exploiting the access patterns of nested patterns
to store sequential memory accesses to on-chip memory and parallelizing the
computation with strip-mining, the compiler can generate hardware that
efficiently utilizes memory bandwidth.  However, the compiler does not
automatically tile patterns for data locality or implement metapipelines
for nested parallel patterns, which we show are essential components for generating efficient hardware. Overall, our work is the first to show a complete method for
automatically tiling parallel patterns to improve locality for individual compute units and a process
for inferring hardware metapipelines from nested parallel patterns.

% We conclude with a qualitative comparison of Spatial to related work, drawing from the criteria in Section~\ref{background}.
A good FPGA design tool should have the following features:
\begin{itemize}
  \item \emph{Representation}: The tool must internally represent hardware using a general and parameterizable
    representation. This representation must preserve information regarding data locality,
    memory access pattern and parallelism in the input at all levels of nesting.
    Such a representation must be target-agnostic and should be targetable from high-level
    language constructs.
  \item \emph{Estimation}: The tool must quickly analyze a design in the above representation
    and estimate metrics such as cycle counts and FPGA resource requirements for a target FPGA.
  \item \emph{DSE}: The tool must be able to leverage the estimators to prune the large design search space,
    walk the space of designs, and find the Pareto-optimal surface.
  \item \emph{Generation}: The tool must be able to automatically generate hardware which can then be
    synthesized and run on the target FPGA. Without this feature, hardware would typically
    be generated using separate toolchains for estimation and generation, which makes
    accurate estimation much harder.
\end{itemize}

Previous work on generating FPGA accelerators has focused on various aspects of the points
mentioned above. Here we provide an overview of this work.

High-level synthesis (HLS) tools such as LegUp~\cite{legup-tecs} and Vivado HLS~\cite{vivadohls} (previously AutoPilot)~\cite{cong11hls}
synthesize hardware from C. These tools provide estimates of the cycle count, area and power consumption
along with hardware generation. However, imperative design descriptions place greater burden on
the compiler to discover parallelism, pipeline structure and memory access patterns.
The absence of explicit parallelism often leads to conservative compiler analyses
producing sub-optimal designs. While some tools allow users to provide compiler
hints in the form of directives or pragmas in the source code, this approach
fails to capture key points in the design space.
\begin{figure}[ht]
  \begin{lstlisting}[mathescape=true, numbers=none, language=C]
L1: for (int i=0; i<R; i++) {
  #pragma HLS PIPELINE II=1
  L11: for (int j=0; j<C; j++) {
    sub[j] = y[i] ? x[i][j]-mu0[j] : x[i][j]-mu1[j];
  }
  L121: for (int j1=0; j1<C; j1++) {
    L122: for (int j2=0; j2<C; j2++) {
      sigma[j1][j2] += sub[j1]*sub[j2];
  }
}
  \end{lstlisting}
  \caption{GDA for high-level synthesis.}
  \label{fig:gda-hls}
\end{figure}
For example, consider Figure~\ref{fig:gda-hls} which represents the gaussian discriminant analysis (GDA)
kernel. All loops in this kernel are parallel loops. One set of valid design points would be to implement $L1$
as a coarse-grained pipeline with $L11$ and $L121$ as its stages. Commercial HLS tools support
 limited coarse-grained pipelining, but with serveral restrictions. For example,
the \emph{DATAFLOW} directive in Vivado HLS enables users to describe coarse-grained pipelines.
However, the directive does not support arbitrarily nested coarse-grained pipelines,
multiple producers and consumers between stages, or coarse-grain pipelining within a finite loop scope~\cite{vivadohls_ug}, as required in the outer loop in Figure~\ref{fig:gda-hls}.
%Such coarse-grained pipelining constructs are unfortunately not supported in most commercial
%HLS tools, leaving these design points unexplored.
%An attempt to pipeline $L1$ would create a large design with a deep pipeline where both $L11$ and $L121$
%are completely unrolled. % \todo{add some numbers}.
%This is a different design point than creating a coarse-grained pipeline.
In addition, compile times for HLS can be long for large designs due to the complications that
arise during scheduling. % \todo{add some numbers}.
Previous studies~\cite{Aladdin} point out other similar issues.
Such limitations restrict the capability of HLS tools to explore more complex design spaces.

Pouchet et al.~\cite{pouchet13fpga}
explore combining HLS with polyhedral analysis to optimize input designs for locality
and use estimates from HLS tools to drive design space exploration. While this captures a larger design
space than previous work by including tile sizes, this approach is limited to the capabilities
of the HLS tools and to benchmarks that have strictly affine data accesses. This paper improves
upon previous work by modeling tiling
parameters in addition to other design points like coarse-grained pipelining of imperfectly nested loops
which are not supported by HLS tools, as well as data-dependent accesses which are not supported by polyhedral analysis.
Chen et al.~\cite{cong_powerdse} describe a simultaneous resource allocation and binding algorithm
and perform design space exploration using a high-level power estimator. They characterize area
usage of primitives and fit linear models to derive estimation
functions. However, this study does not consider higher level design parameters or nested
parallelism as part of the design space. We perform characterization of primitive
operations as well as other coarse-grained templates, which enables us to estimate resource usage for
much more complex accelerators.
CMOST~\cite{cong_cmost} is a C-to-FPGA framework that uses task-level modeling
to exploit multi-level parallelism. While CMOST uses simple analytical models, this paper uses a mixture of
analytical and machine learning models that enables much more fine-grained and accurate estimation of FPGA resource utilization.

Aladdin\cite{Aladdin} is a pre-RTL estimation tool for ASIC accelerator design.
Aladdin uses a dynamic data dependence graph (DDDG) as input and estimates the latency, area, and power
for a variety of designs. However, using a DDDG limits the tool's ability to discover nested parallelism
and infer coarse-grained pipeline structures that require double buffering, especially with complex
memory accesses in patterns like \emph{filters} or \emph{groupBy}s. Also, Aladdin is focused on ASIC designs
while our work focuses on FPGA accelerators which have a different set of challenges, as outlined above.

Other related work~\cite{Deng,Bilavarn,MatchEst,Enzler,Bjureus} explore various ideas from analytical to empirical
models for estimating latency and area of designs in high-level languages. However, these approaches do not
consider complex applications with nested parallelism. Also, previous work either ignores memory or has a relatively
simple model for memory. This paper handles both on-chip and off-chip memory accesses
with varying, data-dependent memory access patterns.


\paragraph{HDLs}
Hardware description languages like Verilog and VHDL are designed for arbitrary circuit description. In order to achieve maximum generality, they require users to explicitly manage timing, control signals, and local memories. Loops are expressed by state machines in flattened RTL.
One exception to this is Bluespec SystemVerilog \cite{bluespec}, which supports state machine inference from nested while loops.
Recent advancements in HDLs have largely been aimed at meta-programming improvements and increasing the size of hardware module libraries.
Languages like Chisel~\cite{chisel}, MyHDL~\cite{myhdl} and VeriScala~\cite{veriscala} make procedural generation of circuits simpler by embedding their HDL in a software language (e.g. Scala or Python). Similarly, Genesis2~\cite{genesis2} adds Perl scripting support to SystemVerilog to help drive procedural generation. While these improvements allow for more powerful meta-programming compared to Verilog \texttt{\small{generate}} statements, users still write programs at a timed circuit level.
%For application accelerators, such a low level of abstraction is often too tedious.


\paragraph{Lime}
Lime is a Java-based programming model and runtime from IBM which aims to provide a single unified language to program heterogeneous architectures. Lime natively supports custom bit precisions and includes collection operations, with parallelism in such operations inferred by the compiler. Coarse-grained pipeline and data parallelism are expressed through ``tasks''. Coarse-grained streaming computation graphs can be constructed using built-in constructs like \texttt{\small{connect}}, \texttt{\small{split}}, and \texttt{\small{join}}. The Lime runtime system handles buffering, partitioning, and scheduling of stream graphs. However, coarse-grained pipelines which deviate from the streaming model are not supported, and the programmer has to use a low-level messaging API to handle coarse-grained graphs with feedback loops. Additionally, the compiler does not perform automatic design tuning. Finally, the compiler's ability to instantiate banked and buffered memories is unclear as details on banking multi-dimensional data structures for arbitrary access patterns are not specified.


\paragraph{HLS}
High-level synthesis tools such as LegUp~\cite{legup}, Vivado HLS~\cite{vivadohls}, Intel's FPGA SDK for OpenCL~\cite{opencl_sdk}, and SDAccel~\cite{sdaccel} allow users to write FPGA designs in C/C++ and OpenCL.
Using these tools, applications can be expressed at a high level, in terms of arrays and untimed, nested loops.
However, while inner loop pipelining, unrolling, and memory banking and buffering are done by the compiler, they generally require explicit user pragmas.
While previous work has used polyhedral tools to automate banking decisions for affine accesses within a single loop nest~\cite{Wang_banking},
it does not address non-affine cases or cases where accesses to the same memory occur in multiple loop nests.
While pragmas like Vivado HLS's \emph{DATAFLOW} enable limited support for pipelining nested loops, pipelining at arbitrary loop nest levels is not yet supported~\cite{vivado_userguide}.
Tools like Aladdin~\cite{Aladdin} have also been created to help automate the process of tuning the pragmas in HLS programs, but designs in HLS still require manual hardware optimization~\cite{nane2016survey}.

%More importantly, users optimizing HLS describe their program in terms of arrays with pragmas.
%While these are software constructs, optimizing hardware accelerators in HLS tools is a
%Numerous issues, including memory aliasing, and low-level software implementation applications.
%While most HLS tools primarily serve as IP core generators, but SDAccel \cite{sdaccel} is a systems-level solution.
%Pragmas allow information to be added back to the compiler, but not in a sound way.  }

\paragraph{MaxJ}
MaxJ is a proprietary language created by Maxeler which allows users to express dataflow algorithms in Java libraries, emphasizing
timing at the level of ``ticks`` of valid streaming elements rather than cycles. ~\cite{maxeler}.
Users must fall back to flattened, HDL-like syntax for state machines when writing nested loops.
Memories are inferred based on relative stream offsets, which, while convenient for stream processing,
hides hardware implementation details from the user which could otherwise help drive optimization.
Additionally, MaxJ has limited portability, as it currently can only be used to target supported Maxeler FPGA platforms.



% Compared to image processing DSLs, Spatial is
% more general but at a lower level of abstraction. Spatial uses a mix of imperative and functional specifications for stencil operations,
% but is able to handle pipelining and unrolling for arbitrary loop hierarchies.
% Spatial requires explicit specification of the memory hierarchy, but is able to automatically bank, buffer, and duplicate for arbitrary access patterns.
% These features, along with Spatial's design space exploration capabilities, make Spatial a natural fit as an optimizing backend target for
% image processing DSLs.
