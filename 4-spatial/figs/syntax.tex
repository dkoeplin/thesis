
\begin{table*}
\centering

%%%%%%%%% ----- Counter ------- %%%%%
\newsavebox{\counter}
\begin{lrbox}{\counter}
\begin{lstlisting}[language=SpatialTable]
min* until max by stride* par p* : Counter
\end{lstlisting}
\end{lrbox}

%%%%%%%%% ----- FSM ------- %%%%%
\newsavebox{\fsmSignature}
\begin{lrbox}{\fsmSignature}
\begin{lstlisting}[language=SpatialTable]
FSM(init){while}{f}{next}: Void
\end{lstlisting}
\end{lrbox}

%%%%%%%%% ----- Foreach ------- %%%%%
\newsavebox{\foreachSignature}
\begin{lrbox}{\foreachSignature}
\begin{lstlisting}[language=SpatialTable]
Foreach(counter+){f}: Void
\end{lstlisting}
\end{lrbox}

%%%%%%%%% ----- Reduce ------- %%%%%
\newsavebox{\reduceSignature}
\begin{lrbox}{\reduceSignature}
\begin{lstlisting}[language=SpatialTable]
Reduce(accum)(counter+){f}{r}: Reg[T]
\end{lstlisting}
\end{lrbox}

%%%%%%%%% ----- MemReduce ------- %%%%%
\newsavebox{\memreduceSignature}
\begin{lrbox}{\memreduceSignature}
\begin{lstlisting}[language=SpatialTable]
MemReduce(accum)(counter+){f}{r}: SRAM$^M$[T]
\end{lstlisting}
\end{lrbox}

%%%%%%%%% ----- Stream ------- %%%%%
\newsavebox{\streamStar}
\begin{lrbox}{\streamStar}
\begin{lstlisting}[language=SpatialTable]
Stream(*){f}: Void
\end{lstlisting}
\end{lrbox}

%%%%%%%%% ----- Parallel ------- %%%%%
\newsavebox{\parallelSignature}
\begin{lrbox}{\parallelSignature}
\begin{lstlisting}[language=SpatialTable]
Parallel{f}: Void
\end{lstlisting}
\end{lrbox}

%%%%%%%%% ----- DummyPipe ------- %%%%%
\newsavebox{\pipeSignature}
\begin{lrbox}{\pipeSignature}
\begin{lstlisting}[language=SpatialTable]
DummyPipe{f}: Void
\end{lstlisting}
\end{lrbox}

%%%%%%%%% ----- IfThenElse ------- %%%%%
\newsavebox{\ifSignature}
\begin{lrbox}{\ifSignature}
\begin{lstlisting}[language=SpatialTable]
if (cond$_1$){f$_1$}
else if (cond$_i$){f$_i$} +
else {f$_N$}* : T
\end{lstlisting}
\end{lrbox}

%%%%%%%%% ----- Sequential Tag ------- %%%%%
\newsavebox{\sequentialTag}
\begin{lrbox}{\sequentialTag}
\begin{lstlisting}[language=SpatialTable]
Sequential.(Foreach|Reduce|MemReduce)
\end{lstlisting}
\end{lrbox}

%%%%%%%%% ----- Pipe Tag ------- %%%%%
\newsavebox{\pipeTag}
\begin{lrbox}{\pipeTag}
\begin{lstlisting}[language=SpatialTable]
Pipe(ii*).(Foreach|Reduce|MemReduce)
\end{lstlisting}
\end{lrbox}

%%%%%%%%% ----- Stream Tag ------- %%%%%
\newsavebox{\streamTag}
\begin{lrbox}{\streamTag}
\begin{lstlisting}[language=SpatialTable]
Stream.(Foreach|Reduce|MemReduce)
\end{lstlisting}
\end{lrbox}

%%%%%%%%% ----- Parallel Tag ------- %%%%%
\newsavebox{\parallelTag}
\begin{lrbox}{\parallelTag}
\begin{lstlisting}[language=SpatialTable]
Parallel.(Foreach|Reduce|MemReduce)
\end{lstlisting}
\end{lrbox}

%%%%%%%%% ----- FIFO ------- %%%%%
\newsavebox{\fifoSyntax}
\begin{lrbox}{\fifoSyntax}
\begin{lstlisting}[language=SpatialTable]
FIFO[T](depth)
\end{lstlisting}
\end{lrbox}

%%%%%%%%% ----- LIFO ------- %%%%%
\newsavebox{\filoSyntax}
\begin{lrbox}{\filoSyntax}
\begin{lstlisting}[language=SpatialTable]
LIFO[T](depth)
\end{lstlisting}
\end{lrbox}

%%%%%%%%% ----- LineBuffer ------- %%%%%
\newsavebox{\lineBufferSyntax}
\begin{lrbox}{\lineBufferSyntax}
\begin{lstlisting}[language=SpatialTable]
LineBuffer[T](r, c)
\end{lstlisting}
\end{lrbox}

%%%%%%%%% ----- LUT ------- %%%%%
\newsavebox{\lutSyntax}
\begin{lrbox}{\lutSyntax}
\begin{lstlisting}[language=SpatialTable]
LUT[T](dim+)(elements+)
\end{lstlisting}
\end{lrbox}

%%%%%%%%% ----- Reg ------- %%%%%
\newsavebox{\regSyntax}
\begin{lrbox}{\regSyntax}
\begin{lstlisting}[language=SpatialTable]
Reg[T](reset*)
\end{lstlisting}
\end{lrbox}

%%%%%%%%% ----- RegFile ------- %%%%%
\newsavebox{\regfileSyntax}
\begin{lrbox}{\regfileSyntax}
\begin{lstlisting}[language=SpatialTable]
RegFile[T](dim+)
\end{lstlisting}
\end{lrbox}

%%%%%%%%% ----- SRAM ------- %%%%%
\newsavebox{\sramSyntax}
\begin{lrbox}{\sramSyntax}
\begin{lstlisting}[language=SpatialTable]
SRAM[T](dim+)
\end{lstlisting}
\end{lrbox}

%%%%%%%%% ----- ArgIn ------- %%%%%
\newsavebox{\argInSyntax}
\begin{lrbox}{\argInSyntax}
\begin{lstlisting}[language=SpatialTable]
ArgIn[T]
\end{lstlisting}
\end{lrbox}

%%%%%%%%% ----- ArgOut ------- %%%%%
\newsavebox{\argOutSyntax}
\begin{lrbox}{\argOutSyntax}
\begin{lstlisting}[language=SpatialTable]
ArgOut[T]
\end{lstlisting}
\end{lrbox}

%%%%%%%%% ----- HostIO ------- %%%%%
\newsavebox{\hostIOSyntax}
\begin{lrbox}{\hostIOSyntax}
\begin{lstlisting}[language=SpatialTable]
HostIO[T]
\end{lstlisting}
\end{lrbox}

%%%%%%%%% ----- DRAM ------- %%%%%
\newsavebox{\dramSyntax}
\begin{lrbox}{\dramSyntax}
\begin{lstlisting}[language=SpatialTable]
DRAM[T](dim+)
\end{lstlisting}
\end{lrbox}

%%%%%%%%% ----- StreamIn ------- %%%%%
\newsavebox{\streamInSyntax}
\begin{lrbox}{\streamInSyntax}
\begin{lstlisting}[language=SpatialTable]
StreamIn[T](bus)
\end{lstlisting}
\end{lrbox}

%%%%%%%%% ----- StreamOut ------- %%%%%
\newsavebox{\streamOutSyntax}
\begin{lrbox}{\streamOutSyntax}
\begin{lstlisting}[language=SpatialTable]
StreamOut[T](bus)
\end{lstlisting}
\end{lrbox}

%%%%%%%%% ----- Parameter ------- %%%%%
\newsavebox{\parameterSyntax}
\begin{lrbox}{\parameterSyntax}
\begin{lstlisting}[language=SpatialTable]
default (min,step*,max)
\end{lstlisting}
\end{lrbox}

%%%%%%%%% ----- Accel ------- %%%%%
\newsavebox{\accelSyntax}
\begin{lrbox}{\accelSyntax}
\begin{lstlisting}[language=SpatialTable]
Accel{body}
\end{lstlisting}
\end{lrbox}

%%%%%%%%% ----- Accel* ------- %%%%%
\newsavebox{\accelStarSyntax}
\begin{lrbox}{\accelStarSyntax}
\begin{lstlisting}[language=SpatialTable]
Accel(*){body}
\end{lstlisting}
\end{lrbox}


\fontsize{8}{10}
\selectfont
\hspace{-10pt}
\begin{tabular}{m{0.45\columnwidth} m{0.01\columnwidth} m{0.45\columnwidth}}
\begin{tabular}{l}

\syntaxtitle{(a) Control Structures}

\multirow{1}{*}{\usebox{\counter}} \\
\vspace{-7pt} \\
A counter over $[\argg{min},\argg{max})$.  \\
\vspace{-8pt} \\
{
\begin{tabular}{lll}
\argg{min}:  &\hspace{-10pt}$\mathbb{Z}$ & \hspace{-4pt}Start value, inclusive. Default is 0. \\
\argg{max}:  &\hspace{-10pt}$\mathbb{Z}$ & \hspace{-4pt}Counter maximum, non-inclusive. \\
\argg{step}: &\hspace{-10pt}$\mathbb{Z}$ & \hspace{-4pt}Counter stride. Default is 1. \\
\argg{p}:    &\hspace{-10pt}$\mathbb{Z}$ & \hspace{-4pt}Counter parallelization. Default is 1. \\
\end{tabular}
} \\ \desyntax\\

% \multirow{3}{*}{\usebox{\ifSignature}} \\ \\ \\
% \vspace{-7pt} \\
% Data-dependent execution of first valid condition. \\
% Doubles as a multiplexer if $T \in \mathbb{B}$. \\
% \vspace{-8pt} \\
% {
% \begin{tabular}{lll}
% \argg{cond$_i$}: &\hspace{-10pt} $\rightarrow$ \scalar{T} & \hspace{-4pt}Condition for associated body. \\
% \argg{f$_i$}: &\hspace{-10pt} $\rightarrow$ \scalar{T} & \hspace{-4pt}Arbitrary expression. \\
% \end{tabular}
% } \\ \desyntax\\

\multirow{1}{*}{\usebox{\fsmSignature}} \\
\vspace{-7pt} \\
An arbitrary finite state machine. \\
Internal state is a value of type $T \in \mathbb{B}$ \\
\vspace{-8pt} \\
{
\begin{tabular}{lll}
\argg{init}:   &\hspace{-10pt} \scalar{T} & \hspace{-4pt}The FSM's initial state.  \\
\argg{while}:  &\hspace{-10pt} \scalar{T} $\rightarrow$ \scalar{Bit}  & \hspace{-4pt}Continue function.  \\
\argg{f}:      &\hspace{-10pt} \scalar{T} $\rightarrow$ \scalar{Void} & \hspace{-4pt}Function executed each iteration.  \\
\argg{next}    &\hspace{-10pt} \scalar{T} $\rightarrow$ \scalar{T}    & \hspace{-4pt}State transition function. \\
\end{tabular}
} \\ \desyntax\\

\multirow{1}{*}{\usebox{\foreachSignature}} \\
\vspace{-7pt} \\
A parallelizable \emph{for} loop. \\
\vspace{-8pt} \\
{
\begin{tabular}{lll}
\argg{counter}: &\hspace{-10pt} \texttt{\textcolor{darkgreen}{Counter$_D$}} & \hspace{-4pt}Loop iteration domain.    \\
\argg{f}:       &\hspace{-10pt} $\mathbb{Z}_D \rightarrow $ \scalar{Void}   & \hspace{-4pt}Loop body. \\
\end{tabular}
} \\ \desyntax\\

\multirow{1}{*}{\usebox{\reduceSignature}} \\
\vspace{-7pt} \\
A reduction of scalars $T \in \mathbb{B}$, parallelized as a tree. \\
\vspace{-8pt} \\
{
\begin{tabular}{lll}
\argg{accum}:   &\hspace{-10pt} \memory{Reg}\texttt{[}\scalar{T}\texttt{]}       & \hspace{-4pt}Accumulation register.    \\
\argg{counter}: &\hspace{-10pt} \memory{Counter$_D$}                             & \hspace{-4pt}Loop iteration domain.    \\
\argg{f}:       &\hspace{-10pt} $\mathbb{Z}_D \rightarrow$ \scalar{T}            & \hspace{-4pt}Map function. \\
\argg{r}:       &\hspace{-10pt} (\scalar{T},\scalar{T}) $\rightarrow$ \scalar{T} & \hspace{-4pt}Combinination function. \\
\end{tabular}
} \\ \desyntax\\

\multirow{1}{*}{\usebox{\memreduceSignature}} \\
\vspace{-7pt} \\
Reduction over $M$-dimensional memories. \\
\vspace{-8pt} \\
{
\begin{tabular}{lll}
\argg{accum}:   &\hspace{-10pt} \memory{SRAM$^M$}\texttt{[}\scalar{T}\texttt{]} & \hspace{-4pt}Accumulation memory.    \\
\argg{counter}: &\hspace{-10pt} \texttt{\textcolor{darkgreen}{Counter$_D$}} & \hspace{-4pt}Loop iteration domain.    \\
\argg{f}:       &\hspace{-10pt} $\mathbb{Z}_D \rightarrow$ \memory{SRAM$^M$}\texttt{[}\scalar{T}\texttt{]}  & \hspace{-4pt}Map function. \\
\argg{r}:       &\hspace{-10pt} (\scalar{T},\scalar{T}) $\rightarrow$ \scalar{T}         & \hspace{-4pt}Combination function. \\
\end{tabular}
} \\ \desyntax\\

\multirow{1}{*}{\usebox{\streamStar}} \\
\vspace{-7pt} \\
A streaming loop which never terminates. \\
\vspace{-8pt} \\
{
\begin{tabular}{lll}
\argg{f}:   &\hspace{-10pt} $\rightarrow$ \scalar{Void} & \hspace{-4pt}Arbitrary expression. \\
\end{tabular}
} \\ \desyntax\\

\multirow{1}{*}{\usebox{\parallelSignature}} \\
\vspace{-7pt} \\
Overrides normal compiler scheduling. All conrollers \\
are instead scheduled in a \emph{fork-join} fashion. \\
\vspace{-8pt} \\
{
\begin{tabular}{lll}
\argg{f}:   &\hspace{-10pt} $\rightarrow$ \scalar{Void} & \hspace{-4pt}Arbitrary sequence of controllers. \\
\end{tabular}
} \\ \\

% \multirow{1}{*}{\usebox{\pipeSignature}} \\
% \vspace{-7pt} \\
% A ``loop'' with exactly one iteration. \\
% Inserted by the compiler, generally not written explicitly. \\
% \vspace{-8pt} \\
% {
% \begin{tabular}{lll}
% \argg{f}:   &\hspace{-10pt} $\rightarrow$ \scalar{Void} & \hspace{-4pt}Arbitrary expression. \\
% \end{tabular}
% } \\ \desyntax\\

\syntaxtitle{(b) Optional Scheduling Directives}

\multirow{1}{*}{\usebox{\sequentialTag}} \\
\vspace{-7pt} \\
Sets loop to run sequentially. \\
\desyntax\\

\multirow{1}{*}{\usebox{\pipeTag}} \\
\vspace{-7pt} \\
Sets loop to be pipelined. \\
{
\begin{tabular}{lll}
\argg{ii}:   &\hspace{-10pt} $\mathbb{Z}$ & \hspace{-4pt}Optional initiation interval . \\
\end{tabular}
}
\\ \desyntax\\

\multirow{1}{*}{\usebox{\streamTag}} \\
\vspace{-7pt} \\
Sets loop to be streaming. \\

\end{tabular} & &
\begin{tabular}{l}

\syntaxtitle{(c) On-Chip Memories}

\multirow{1}{*}{\usebox{\fifoSyntax}} \\
\vspace{-7pt} \\
FIFO with \argg{depth} elements of type \scalar{T}. \\
{
\begin{tabular}{lll}
\argg{depth}: &\hspace{-10pt} $\mathbb{Z}$ & \hspace{-4pt}Capacity of FIFO. \\
\end{tabular}
}
\\ \desyntax\\
%
% \multirow{1}{*}{\usebox{\filoSyntax}} \\
% A LIFO (stack) with a capacity of \argg{depth} elements of type \argg{T} \\
% \desyntax \\

% \multirow{1}{*}{\usebox{\lineBufferSyntax}} \\
% \vspace{-7pt} \\
% Buffer containing \argg{r} ``lines'' of \argg{c} elements. \\
% {
% \begin{tabular}{lll}
% \argg{r}: &\hspace{-10pt} $\mathbb{Z}$ & \hspace{-4pt}Number of lines. \\
% \argg{c}: &\hspace{-10pt} $\mathbb{Z}$ & \hspace{-4pt}Elements per line. \\
% \end{tabular}
% }
% \\ \desyntax\\
%
% \multirow{1}{*}{\usebox{\lutSyntax}} \\
% Read-only Lookup Table containing supplied \argg{elements} of type \argg{T} \\
% \vspace{-10pt}\\

\multirow{1}{*}{\usebox{\regSyntax}} \\
\vspace{-7pt} \\
Register holding a value of type \argg{T}. \\
{
\begin{tabular}{lll}
\argg{reset}: &\hspace{-10pt} \scalar{T} & \hspace{-4pt}Optional reset value. \\
\end{tabular}
}
\\ \desyntax \\

\multirow{1}{*}{\usebox{\regfileSyntax}} \\
\vspace{-7pt} \\
$D$-dimensional register file of elements of type \scalar{T}.\\
{
\begin{tabular}{lll}
\argg{dim}: &\hspace{-10pt} $\mathbb{Z}_D$ & \hspace{-4pt}Memory dimensions. \\
\end{tabular}
}
\\ \desyntax \\

\multirow{1}{*}{\usebox{\sramSyntax}} \\
\vspace{-7pt} \\
$D$-dimensional scratchpad of elements of type \scalar{T}. \\
{
\begin{tabular}{lll}
\argg{dim}: &\hspace{-10pt} $\mathbb{Z}_D$ & \hspace{-4pt}Memory dimensions. \\
\end{tabular}
}
\\ \\


\syntaxtitle{(d) Shared Host/Accelerator Memories}

\multirow{1}{*}{\usebox{\argInSyntax}} \\
\vspace{-7pt} \\
Accelerator register initialized by the host. \\
\desyntax\\

\multirow{1}{*}{\usebox{\argOutSyntax}} \\
\vspace{-7pt} \\
Accelerator register readable by the host. \\
\desyntax\\

\multirow{1}{*}{\usebox{\hostIOSyntax}} \\
\vspace{-7pt} \\
Accelerator register the host may access at any time. \\
\desyntax\\

\multirow{1}{*}{\usebox{\dramSyntax}} \\
\vspace{-7pt} \\
Burst-addressable, host-allocated off-chip memory. \\
{\begin{tabular}{lll}
\argg{dim}:   &\hspace{-10pt} $\mathbb{Z}_D$ & \hspace{-4pt}Memory dimensions. \\
\end{tabular}
}
\\ \\


\syntaxtitle{(e) External Interfaces}

\multirow{1}{*}{\usebox{\streamInSyntax}} \\
\vspace{-7pt} \\
Streaming input from a \argg{bus} of external pins. \\
\desyntax\\

\multirow{1}{*}{\usebox{\streamOutSyntax}} \\
\vspace{-7pt} \\
Streaming output to a \argg{bus} of external pins. \\
\\

\syntaxtitle{(f) Host Interfaces}

\multirow{1}{*}{\usebox{\accelSyntax}} \\
\vspace{-7pt} \\
A blocking accelerator design. \\
\desyntax \\

\multirow{1}{*}{\usebox{\accelStarSyntax}} \\
\vspace{-7pt} \\
A non-blocking accelerator design. \\
\\


\syntaxtitle{(g) Design Space Parameters}

\multirow{1}{*}{\usebox{\parameterSyntax}} \\
\vspace{-7pt} \\
A compiler-aware design parameter. \\
DSE explores the range [\argg{min}, \argg{max}] with optional step. \\
{\begin{tabular}{lll}
\argg{default}: &\hspace{-10pt} $\mathbb{Z}$ & \hspace{-4pt}Value when DSE is not run. \\
\argg{min}:   &\hspace{-10pt} $\mathbb{Z}$ & \hspace{-4pt}DSE space minimum value. \\
\argg{step}:  &\hspace{-10pt} $\mathbb{Z}$ & \hspace{-4pt}DSE space optional step size. \\
\argg{max}:   &\hspace{-10pt} $\mathbb{Z}$ & \hspace{-4pt}DSE space maximum value. \\
\end{tabular}
}

\end{tabular}
\end{tabular}

\caption{A subset of Spatial's syntax.
%An overview of Spatial's syntax for host interfaces, control structures, scheduling directives, memory templates, streaming interfaces, and design space parameters.
Square brackets (e.g. \texttt{[T]}) represent a template's type parameter. Parameters followed by a `\texttt{+}' denote arguments which can be given one or more times, while a `\texttt{*}' denotes that an argument is optional.
%\texttt{DRAMs}, \texttt{Foreach}, \texttt{Reduce}, and \texttt{MemReduce} can all have arbitrary dimensions.
%\texttt{DRAMs} can be allocated with an arbitrary number of dimensions. \texttt{Foreach}, \texttt{Reduce}, and \texttt{MemReduce} support multi-dimensional iteration domains.
}
\label{t:syntaxTable}
\end{table*}
