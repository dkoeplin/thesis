\chapter{Conclusions}
\label{conclusion}
With the end of Dennard scaling and the slow death of Moore's Law, the period
of ``free'' performance improvements on conventional CPU architectures is coming
to a close. As it does, computer architects and computer scientists alike are
looking to more specialized hardware architectures to continue improving
runtimes, throughput, and energy efficiency for performance critical applications.

While reconfigurable architectures like FPGAs are a natural fit for these
specialized hardware designs, their adoption has been historically limited by
their low level programming models. While ``C-to-gates'' style
high level synthesis tools have sought to fill this gap,
their ad-hoc mixture between hardware and software have made them
difficult to adopt when implementing optimized hardware designs.

Meanwhile, as performance requirements have been increasing, the demands for
improvements in abstractions in programming languages have also been increasing.
This is most clear in data analytics, and particularly in machine learning, where
new domain-specific languages are consistently being developed and improved.

If both of these trends continue, it is likely that this will result in a situation
where huge amounts of performance-critical code is being initially written and
tested in high level, hardware target agnostic domain specific languages.
This provides a perfect opportunity for compiler development.

In this dissertation, we describe a system which compiles high level domain
specific languages to reconfigurable architectures, with a particular focus on FPGAs.
We described several high level transformations required on a parallel pattern IR
in order to prepare the representation for hardware-specific lowering.
We then discussed the limitations with lowering to existing high level synthesis abstractions,
both in terms of their slow compile times - limiting design tuning opportunities - and their
inability to exploit arbitrary levels of nested parallelism to achieve better performance.
We instead proposed a new hardware abstraction for our system called Spatial,
which is designed from the ground up to serve as an intermediary between
high level languages and low level hardware designs.
We showed that, using this new abstraction, our system can
tune explore design parameter spaces 279 to 6533 times
faster than a commercial high-level synthesis tool, and produce designs with average speedups
of $2.9\times$ over SDAccel.
The Spatial abstraction is also coupled with a frontend language which is more productive for power users,
which we showed to have an average of 42\% fewer lines of code when defining the same applications.

Going forward, there are still a number of open research directions for this system.
While we evaluated applications directly written in Spatial to provide a fair
comparison to HLS tools at roughly the same level of abstraction, no study has yet
been done on the source or degree of overheads of automatically produced designs
from high level DSLs. Further investigation here would ultimately yield a more robust
and tuned lowering process from higher level languages.

The current Spatial compiler has a number of areas where it can be further expanded and
improved. The memory banking analysis pass, for example, is generally fast, but has
an extremely long worst case running time. This comes down to its reliance on
iterative conflict polytope emptiness testing~\cite{Wang_banking}. Current work
at Stanford is looking at how to use heuristics to speed this up, but more theoretical
work on a faster solver for the conflict polytope could improve the worst case running time.

The design tuning strategies discussed in Section~\ref{dse} show the potential for
practical application of other design space exploration or hyper-parameter tuning
strategies. While the integration with HyperMapper presented here was a preliminary
evaluation, further work evaluating HyperMapper more in depth against random search
with heuristic pruning is ongoing.

The growing adoption of domain specific languages and the growing need for hardware
acclerators presents a huge opportunity for compiler development. The target-agnostic
nature of DSLs, combined with the huge design spaces of hardware accelerator designs,
presents a great degree of freedom for optimizations and program transformations
for compilers. Our system shows how these opportunities can be taken advantage of
in order to both present users with an extremely high level programming model while
also producing performant hardware accelerator designs.

The Spatial language and compiler is an ongoing, open source project at Stanford.
Other related documentation, publications, and code releases can be found
at \url{https://spatial.stanford.edu}.
