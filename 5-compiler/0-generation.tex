\section{Hardware Generation}
\label{hardware}

In this section, we describe how the tiled intermediate representation can be
lowered into the Spatial hardware abstraction described in Chapter~\ref{spatial}.
This step forms the logical boundary between the higher level compiler and the Spatial hardware compiler.

\subsection{Memory Allocation}
Generating efficient FPGA hardware requires effective usage of on-chip memories (buffers).
Here, we allocate buffers for arrays based on data access patterns and size.
%Since the maximum size of data inputs can not usually be inferred statically, we support the use of programmer hints about tile sizes to aid this analysis.
%Information from these annotations is propagated through the IR during this analysis.
All arrays with statically known sizes, such as array \emph{copies} generated in the tiling transformation described in
Section~\ref{sec:strip-mining}, are assigned to buffers.
Dynamically sized arrays are kept in main memory and we generate
caches or buffers with gather operations for any non-affine accesses to these arrays.

Special memories are used for buffers which have unique access patterns.
Streaming or ``one-touch'' data is  either streamed directly to and from off-chip memory
or realized using FIFOs on-chip depending on their usage.
% The tiling transformations described in Section
% represent data tile copies with explicit nodes in the IR. This greatly simplifies memory allocation analysis.
% to first determine where data
% is stored and how it is accessed.
% The goal of this analysis is to assign buffers
% to either on-chip or off-chip memories and select a memory template based on the properties
% of the buffer.
%In general, we use the following heuristics to allocate memories:
%\begin{itemize}
%  \item Buffers which have statically unknown dimensions (typically inputs and outputs)
%    are allocated off-chip. However, we allow programmers to provide hints to the compiler using
%    annotations which can enable more aggressive compiler optimizations. For example, the programmer
%    can hint that a certain buffer is ``cacheable'' -- meaning that the size of the buffer is typically
%    small enough (a few Kilobytes) to completely fit on-chip. Our analysis makes use of this information
%    and performs more aggressive on-chip memory allocation.
%  \item Buffers corresponding to tiles are allocated on-chip.
%  \item
%  \item Intermediate buffers in a \emph{metapipeline} (described in the next section) are realized
%    using \emph{Double Buffers}.
%\end{itemize}

%While the memory allocation analysis performs an initial assignment of templates to buffers, certain special
%buffers need to be \emph{promoted} to use the double buffer template. This is determined during
%metapipeline analysis, described in the next section.


%\subsection{Pipeline Execution Units}
\subsection{Pipeline Execution Units}
We generate parallelized and pipelined hardware when parallel patterns compute with scalar values,
as occurs for the innermost patterns.
We implemented templates for each
pipelined execution unit in Table~\ref{t-hwtemplates} using MaxJ language
constructs, and instantiate each template with the proper parameters (e.g., data type,
vector length) associated with the parallel pattern.  The MaxJ compiler
applies low-level hardware optimizations such as vectorization, code
scheduling, and fine-grained pipelining, and generates efficient hardware.  For
example, we instantiate a reduction tree for a MultiFold over an
array of scalar values, which is automatically pipelined by the MaxJ compiler.



%\subsection{Metapipelining}
