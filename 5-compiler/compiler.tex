\chapter{The Spatial Compiler}
\label{compiler}

Now that we have defined our input representation (parallel patterns) and given
an overview of our Spatial hardware abstraction, we next describe the lowering
process and the details of the Spatial compiler. Since Spatial is designed to
to target reconfigurable architectures like FPGAs, there are a
number of analyses and optimizations that differ from standard software compilers.
The compiler's key passes are summarized in Figure~\ref{fig:spatial-diag} and
described in this chapter.

\begin{figure*}
\centering
\includegraphics[width=0.8\textwidth]{5-compiler/figs/spatial-diag}
\caption{\label{fig:spatial-diag}Block diagram of the Spatial compiler.
This compiler optimizes the Spatial hardware IR for a selected FPGA target
and generates synthesizable Chisel RTL.}
\end{figure*}

The Spatial compiler provides translations from applications or IR
in the Spatial abstraction to synthesizable hardware descriptions in Chisel RTL~\cite{chisel}.
The input to Spatial can be either in the form of the accompanying Spatial language for ``power'' users,
or in-memory tiled parallel patterns when compiling from higher level DSLs.
The generated Chisel RTL is then compiled by the third party Chisel compiler and used
to generate Verilog, which can in turn be synthesized on an FPGA target using vendor tools.
Spatial also generates host code in C++ for FPGA runtime administration and data management.



\section{Intermediate Representation}

Spatial programs are internally represented in the compiler as a hierarchical dataflow graph (DFG).
Nodes in this graph represent control structures, data operations, and memory allocations, while edges represent data and effect dependencies.
Nesting of controllers directly translates to the hierarchy in the intermediate representation.
Design parameters are kept as graph metadata, such that they can be independently updated without changing the graph itself.

\newsavebox{\gemm}
\begin{lrbox}{\gemm}
\begin{lstlisting}[language=Spatial,linewidth=0.4\textwidth]
// Load data from files
val a = loadMatrix[Float](args(0))
val b = loadMatrix[Float](args(1))

// Allocate space on accelerator DRAM
val A = DRAM[Float](a.rows,a.cols)
val B = DRAM[Float](b.rows,b.cols)
val C = DRAM[Float](a.rows,b.cols)

// Create explicit design parameters
val M = 128 (64, 1024)  // Tile size - rows
val N = 128 (64, 1024)  // Tile size - cols
val P = 128 (64, 1024)  // Tile size - common
val PAR_K  = 2 (1, 8)   // Unroll factor of k
val PAR_J  = 2 (1, 16)  // Unroll factor of j

// Transfer data to accelerator DRAM
sendMatrix(A, a)
sendMatrix(B, b)

// Specify the accelerator design
Accel {
  // Produce C in M x N tiles
  Foreach(A.rows by M, B.cols by N){
    (ii,jj) =>
    val tileC = SRAM[Float](M, N)
    // Combine intermediates (outer)
    MemReduce(tileC)(A.cols by P){ kk =>
      // Allocate on-chip scratchpads
      val tileA = SRAM[Float](M, P)
      val tileB = SRAM[Float](P, N)
      val accum = SRAM[Float](M, N)

      // Load tiles of A and B from DRAM
      tileA load A(ii::ii+M, kk::kk+P)
      tileB load B(kk::kk+P, jj::jj+N)

      // Combine intermediates (across P)
      MemReduce(accum)(P par PAR_K){ k =>
        val partC = SRAM[Float](M, N)
        Foreach(M by 1, N par PAR_J){ (i,j) =>
          partC(i,j) = tileA(i,k) * tileB(k,j)
        }
        partC
      // Combine with element-wise add
      }{(a,b) => a + b }
    }{(a,b) => a + b }

    // Store the tile of C to DRAM
    C(ii::ii+M, jj::jj+N) store tileC
  }
}
// Save the result to another file
saveMatrix(args(2), getMatrix(C))
\end{lstlisting}
\end{lrbox}

\begin{figure}
\begin{tabular}{cm{0.4\textwidth}m{0.6\textwidth}}
{\hspace{10pt}\usebox{\gemm}} &
{\includegraphics[width=0.55\textwidth]{5-compiler/figs/ctrltree.pdf}} \\
{\parbox{0.4\textwidth}{\centering{a. Spatial implementation. }}} &
{\parbox{0.6\textwidth}{\centering{b. Control/access tree for \texttt{\small{tileB}}.}}}
\end{tabular} %% end resizebox
\caption{Matrix multiplication ($C$~=~$A~\cdot~B$) implemented in Spatial and corresponding control/access tree IR. Control nodes are annotated with their level (outer versus inner), schedule, and loop iterator name. Memory access nodes are annotated with their parallelization factor.}
\label{fig:matmult}
\end{figure}


When discussing DFG transformations and optimizations, it is often useful to think about the graph as a controller/access tree. Figure~\ref{fig:matmult} shows an example of one such controller tree for the memory {\texttt{\small{tileB}} in the Spatial code example of matrix multiplication. Note that transfers between on-chip and off-chip memory each expand to a control node which linearly accesses the on-chip memory, in this case by iterators \texttt{e} and \texttt{f} which have corresponding parallelization factors
\texttt{PAR\_E} and \texttt{PAR\_F}.
This tree abstracts away most primitive operations, leaving only relevant controller hierarchy and the memory
accesses for a specific memory.

Within the acceleratable subset of Spatial, nodes are formally separated into three categories:
control nodes, memory allocation nodes, and primitive nodes.
Control nodes represent state machine structures like \texttt{\small{Foreach}} and \texttt{\small{Reduce}} described in Section~\ref{controls}.
Primitive nodes are operations which may consume, but never produce, control signals, including on-chip memory accesses.
Primitive nodes are further broken down into ``physical'' operations requiring resources and ``ephemeral'' operations which are only used for bookkeeping purposes in the compiler. For example, bit selects and grouping of words into structs require no hardware resources but are used to track necessary wires in the generated code.



% The nodes that compose the IR of Spatial provide the handles necessary to do a range of
% hardware optimizations that are specific to spatial architectures.  The combination of
% metadata associated with each node and the hierarchical structure AST that exposes relationships
% between primitives and control structures make it easy to do optimizations on the scheduling of
% controllers, buffering, banking, and duplication of memory elements, and comprehensive DSE over
% the provided parameter space with low latency.
% \subsection{DRAM Request Consolidation}
% In memory-bound applications, the only way to improve performance is to make better use
% of the available bandwidth.  It is well known that memory bandwidth asymptotically approaches the DRAM's peak bandwidth \todo{is this true?}
% as the size of each request increases.  This is because of how DRAM pays a penalty for activating and retiring
% lines of memory cells, and can return more data quickly when consecutive bursts are requested with the same command.

% Unfortunately, there are many applications where the programmer may opt to create logical tensors with
% relatively small leading dimensions and attempt to load multi-dimensional portions of the structure into on-chip SRAM
% without awareness of how this may thrash the DRAM's controllers in an inefficient way.  For example,
% the programmer may want to solve a multi-objective gradient descent problem that has many training points and very
% few objectives, hence creating a tall and skinny Y matrix.

% The compiler is able to recognize when the application will be sending out multiple requests to DRAM with
% consecutive addresses, and rewrite the controller to consolidate these into fewer, longer burst commands.
% This means that the user will get fully optimized DRAM requests and physical hardware without needing
% to rethink or change the semantics of the source code.

\section{Control Insertion}

To simplify reasoning about control signals, Spatial requires that control nodes do not contain both physical primitive nodes and other control nodes. The exception to this rule is conditional \texttt{\small{if}} statements, which can be used in the same scope as primitives as long as they contain no control nodes but conditionals themselves.
This requirement is satisfied by a DFG transformation which inserts \texttt{\small{DummyPipe}} control nodes around primitive logic in control bodies which also contain control nodes. The \texttt{\small{DummyPipe}} node is a bookkeeping control structure which is logically equivalent to a loop with exactly one iteration.
Thereafter, control nodes with primitive nodes are called ``inner'' control nodes, while controllers which contain other nested controllers are called ``outer'' nodes.

% For example, Figure~\ref{fig:matmult} contains some of these nodes. The Foreach in line 32 is an ``outer'' controller, which contains a memory allocation node for tileC
% (line 33) and another control node, \texttt{\small{MemReduce}} in line 38.  The \texttt{\small{Foreach}} in line 51 is an ``inner'' controller, as it contains
% only primitive nodes generated by the SRAM reads, multiplication, and SRAM store inlined on line 52.


\section{Controller Scheduling}
\label{scheduling}
After controller insertion, the compiler will then schedule the operations within each controller.
By default, the compiler will always attempt to pipeline loops regardless of nesting level.
The behavior of the compiler's scheduler can be overridden by the user using the directives listed in Table~\ref{t:syntaxTable}b.

Inner pipeline schedules are based on their initiation interval.
The compiler first collects resource initiation intervals for each primitive node in the given controller based on an internal, target-dependent lookup table.
Most primitive operations are pipelined for a resource initiation interval of 1.
The compiler then calculates all loop carried dependencies within the pipeline based on the dataflow graph.
For non-addressable memories, the total initiation interval is the maximum of path lengths between all dependent reads and the writes.
For addressable memories, the path length of loop carried dependencies is also multiplied by the difference in write and read addresses.
If the addresses are loop-independent, the initiation interval is the path length if they may be equal, and 1 if they are provably not equal. If the distance between the addresses cannot be determined statically, the initiation interval is infinite, meaning the loop must be run sequentially.
The total initiation interval is defined as the maximum of the initiation intervals of all loop carried dependencies and all resource initiation intervals.

The compiler also attempts to pipeline the bodies of outer control nodes in a similar manner, but computes dataflow scheduling in terms of inner control nodes and number of stages rather than primitive nodes and cycles. For example, the outer \texttt{\small{MemReduce}} in line 28 of Figure~\ref{fig:matmult} contains 4 sub-controllers: the load into \texttt{\small{tileA}} (line 35), the load into \texttt{\small{tileB}} (36), the inner \texttt{\small{MemReduce}} (39), and a reduction stage combining intermediate tiles (47). Based on data dependencies, the compiler infers that the two loads can be run in parallel, followed by the inner \texttt{\small{MemReduce}} and the tile reduction. From the information provided
from the higher level compiler or data dependency analysis, it can also determine
that multiple iterations of this outer loop can be pipelined through these stages.

\input{5-compiler/2.1-memory-analysis}
\section{Area and Runtime Modeling}
\label{sec:modeling}
In this section, we describe our area and runtime modeling methodology. Our models account for the various
design parameters for each Spatial node, as well as optimizations done by low-level logic
synthesis tools in order to accurately estimate resource usage.

\subsection{Modeling Considerations}
\label{ss:modeling-con}
The resource requirements of a given application implemented on an
FPGA depend both on the target device and on the toolchain.
Heterogeneity in the FPGA fabric, use of FPGA resources for routing,
and other low-level optimizations performed by logic synthesis tools often have
a significant impact on the total resource consumption of a design.
Since these factors reflect the physical layout of computation on the device
after placement and routing, they are not captured directly in the application's dataflow graph.
We identify and account for the following factors:

\paragraph{LUT and register packing} Basic compute units in FPGAs are typically composed of a lookup table (LUT), and a small number of single bit multiplexers, registers, and full adders.
  Modern FPGA LUTs support up to 8-input binary functions but are often implemented using a pair of smaller LUTs~\cite{stratixv,virtex7}.
  When these LUTs can be configured and used independently, vendor placement tools attempt to ``pack'' multiple small functions into a single 8-input unit.
  LUT packing can have a significant impact on design resource requirements.
  In our experiments, we are able to pack about 80\% of the functions in each design in pairs, decreasing the number of used LUTs by about 40\%.

\paragraph{Routing Resources} Logic synthesis tools require a significant amount of resources to establish static routing connections
between two design points (e.g., a multiplier and a block RAM) which fit in the path's clock period. While FPGAs have dedicated routing resources,
logic synthesis tools may have the option to use LUTs for routing. These LUTs may then be unavailable to be used for ``real'' compute.
In our designs, ``route-through'' LUTs typically account for about 10\% of the total number of used LUTs.


\paragraph{Logic duplication} Logic synthesis tools often duplicate resources such as block RAMs
and registers to avoid routing congestion and to decrease fan out. While duplicated registers typically
encompass around 5\% of the total number of registers required in our designs, we found that block RAM
duplication can increase RAM utilization by 10 to 100\%, depending on the complexity of the design.


\paragraph{Unavailable resources} FPGA resources are typically organized in a hierarchy, such as Altera's Logic Array Block structure (10 LUTs)
and Xilinx's Slice structure (4 LUTs). Such organizations impose mapping constraints which can lead to
resources that are rendered unusable. In our experiments, the number of unusable LUTs made up only about 4\% of the design's total LUT usage.

\subsection{Methodology}
In order to model runtime and resource requirements of Spatial designs, we first need an estimate of the area requirements and
propagation delay of every Spatial node. Area requirements include the number of digital signal processing
units (DSPs), device block RAMs, LUTs, and registers that each template requires. To facilitate LUT packing estimation,
we split template LUT resource requirements into the number of ``packable'' and ``unpackable'' LUTs required.
%For templates with documented implementations such as floating point operations \cite{fp-altera}, we gather resource and cycle data in part by summarizing IP core documentation. However, these user manuals typically do not give resource usage breakdowns at the binary function level.
We obtain characterization data by synthesizing multiple instances of each template instantiated for combinations of its parameters as given in Table~\ref{t-hwtemplates}.
Using this data, we create analytical models of each Spatial nodes's resource requirements and cycle counts for
a predefined fabric clock. The area and cycle count of controller templates are modeled as functions of the latencies of the nodes contained within them.
The total cycle count for a coarse-grain pipelined controller, for example,
is modeled using the recursive function
\begin{displaymath}
(N-1)\max(cycles(n) | n \in nodes) + \sum_{n \in nodes} cycles(n)
\end{displaymath}
where \emph{N} is the number of iterations of the controller and \emph{nodes} is the
longest chain of dependencies within the set of children
controllers contained in the coarse-grain pipelined controller.

Most templates require about six synthesized designs to characterize their resource and area usage as a function of their parameters. Note that these models include estimates of off-chip memory access latency as a function of the
number and length of memory commands, as well as contention due to competing accessors. Since template models are application-independent, each needs only be characterized once for a given target device and logic synthesis toolchain. The synthesis times required to model templates can therefore be amortized over many applications.

Using these models, we run a pair of analysis passes over the application's intermediate representation to estimate design cycle counts and area requirements.

\subsubsection{Cycle Count Estimation}
In the first analysis pass, we estimate the total runtime of the design on the FPGA.
Since the Spatial intermediate representation is hierarchical in nature, this pass is done recursively.
The total runtime of \emph{MetaPipe} and \emph{Sequential} nodes is calculated first by determining
the runtime of all controller nodes contained within them. The total propagation delay of a single
iteration of a \emph{Pipe} is the length of the body's critical path, calculated using a depth first
search of the body's subgraph and the propagation delay of all primitive nodes within the graph.
Input dataset sizes, given as user annotations in the high-level program, are used by the analysis pass
along with tiling factors to determine the iteration counts for each controller template.
Iteration counts are then used to calculate the total runtime of the respective controller nodes.


\subsubsection{Area Estimation}
Since the FPGA resource utilization of a design is sensitive to factors that are not directly captured
in the design's dataflow graph, we adopt a hybrid approach in our area analysis.

We first estimate the area of the design by counting the resource
requirements of each node using their pre-characterized area models. In pipelined controller bodies, we also
estimate the resources required for delaying signals. This is done by recursively calculating the
propagation delay of every path to each node using depth first search. Paths with slack relative to
the critical path to that node require their width (in bits) multiplied by the slack delay resources.
Delays over a synthesis tool-specific threshold are modeled as block RAMs.
Otherwise, they are modeled as registers. Note that this estimation assumes ASAP scheduling.

We model LUT routing usage, register duplication, and unavailable LUTs using a set of small artificial
neural networks implemented using the Encog machine learning library~\cite{encog}.
Each network has three fully connected layers with eleven input nodes, six hidden layer nodes, and a single output node. We chose to use
three layer neural networks as they have been proven to be capable of fitting a wide number of function classes with arbitrary precision, including polynomial functions of any order~\cite{science}.
One network is trained
for each factor on a common set of 200 design samples with varying levels of resource usage to give a representative sampling of the space. Choosing the correct
network parameters to obtain the lowest model error is typically challenging, but in our experiments we found that above four nodes in the hidden layer,
the exact number of hidden layer nodes made little difference.
Duplicated block RAMs are estimated as a linear function of the number of routing LUTs, as we found that this gave the best estimate of design routing complexity in practice. This linear function was fit using the same data used to train the neural networks. Like the template models, these neural networks are application independent and only need to be trained once for a given target device and toolchain.

We use the raw resource counts as an input to each of our neural networks to obtain global estimates for routing LUTs,
duplicated registers, and unavailable LUTs. We estimate the number of duplicated block RAMs using the routing LUTs.
These estimates are then added to the raw resource counts to obtain a pre-packing resource estimate. For the purposes
of LUT packing, we assume routing LUTs are always packable.

Lastly, we model LUT packing using the simple assumption that all packable LUTs will be packed. The target device
in our experiments supports pairwise LUT packing, so we estimate the number of compute units used for logic as
the number of unpackable LUTs plus the number of packable LUTs divided by two. We assume that each compute unit
will use two registers on average. We model any registers unaccounted for by logic compute units as requiring
compute units with two registers each. This gives us the final estimation for LUTs, DSPs, and BRAM.

%We use a simple model for LUT packing, where we assume that pairs of LUTs which
%can be packed are packed.
%Table \todo{Table with control, memory, and mem generator templates as functions + description of what each variable is - should have four columns(?): node name, area function, and cycle function, and variable description?} gives examples of the resource usage and cycle latency of memory, controller, and memory command generator templates as functions of their respective parameters.
%We model each of the factors listed in section~\ref{ss:modeling-con} using a two layer artificial neural networks

%We use the Encog machine learning library~\cite{encog} to implement our neural network.

\section{Design Parameter Tuning}
\label{dse}

The scheduling and memory banking options identified by the compiler, together with loop parallelization and tile size parameters, forms a design space for the application.
The design tuning pass is an optional compiler pass which allows for fast exploration of this design space in order to make area/runtime design tradeoffs.
When design tuning is enabled, it repeatedly picks design points and evaluates them by rerunning the control scheduling, memory analysis, and estimation analysis passes. The output from this search is a single set of parameters from the Pareto frontier.

Unfortunately, application design spaces tend to be extremely large, and exhaustive search on an entire space is often infeasible. Of the benchmarks we evaluate in Chapter\ref{spatial-evaluation},
only BlackScholes has a relatively small space of about $80,000$ points. While this space can be explored exhaustively by Spatial in a few minutes, other spaces are much larger, spanning $10^6$ to $10^{10}$ points and
taking hours or days to exhaustively search. For example, even with the few explicit design parameters exposed in the code in Figure~\ref{fig:matmult}, when combined with implicit pipelining and parallelization parameters, this code already has about $2.6\times10^8$ potential designs.

\subsection{Heuristic Random Search}
In the first implementation of our design tuning pass, we employ purely
random search with heuristic pruning with a fixed number of selected design points.
As we are dealing with large design spaces on the order of millions of points even for small benchmarks,
we prune invalid and suboptimal points in the search space using a few simple heuristics:
\begin{itemize}
  \item Parallelization factors considered are integer divisors of the respective iteration counts. We use this pruning strategy because non-divisor factors create edge cases which require additional modulus operations. These operations can significantly increase the latency and area of address calculation, typically making them poor design parameter choices \cite{raghus-paper}.
  \item Tile sizes considered are divisors of the dimensions of the annotated data size. Similar to parallelization factors, tile sizes with edge cases are usually suboptimal as they increase load and store area and latency with additional indexing logic.
  \item Automatic banking of on-chip memories eliminates the memory banks as an independent variable. This prunes a large set of suboptimal design points where on-chip memory bandwidth requirements do not match the amount of parallelization.
  \item The total size of each local memory is limited to a fixed maximum value.
\end{itemize}

These heuristics defines a ``legal'' subspace of the total design space and can
generally reduce the total design space by two or three orders of magnitude.
In our experiments, we randomly generate estimates for up to $75,000$ legal
points to give a representative view of the entire design space. We immediately
discard illegal points. However, this approach gave relatively high variance
on larger design spaces and has no guarantee against inadvertently skipping desirable points.

\subsection{Active Learning Based Search}
To reduce the variance on larger design spaces, the second version of
Spatial's design space exploration flow integrates an active learning-based autotuner called HyperMapper~\cite{Bodin2016:PACT16,NardiBSVDK17,Saeedi_ICRA_2017}.
HyperMapper is a multi-objective derivative-free optimizer (DFO), and has already been demonstrated on the SLAMBench benchmarking framework \cite{nardi2015introducing}.
HyperMapper creates a surrogate model using a random forest regressor,
and predicts the performance over the parameter space.
This regressor is initially built using only few hundred random design point
samples, as compared to the $75,000$ samples in random search,
and is iteratively refined in subsequent active learning steps. This approach
forces the search process to focus only on points which are likely to be closer to the
Pareto as the search progresses, meaning the search should take far fewer total
points to approximate the true design Pareto.


\subsection{Unrolling}
Following selection of values for design parameters, Spatial finalizes these parameters in a single graph transformation which unrolls loops and duplicates memories as determined by prior analysis passes.
\texttt{Reduce} and \texttt{MemReduce} patterns are also lowered into their imperative implementations, with hardware reduction trees instantiated from the given reduction function.
The two \texttt{MemReduce} loops in Figure~\ref{fig:matmult}, for example, will each be lowered into unrolled \texttt{Foreach} loops with explicitly banked memory accesses and explicitly duplicated multiply operations. The corresponding reduction across tiles (lines 46~--~47) are lowered into a second stage of the \texttt{Foreach} with explicit reduction trees matching the loop parallelization.

\subsection{Retiming}
After unrolling, the compiler retimes each inner pipeline to make sure data and control signals properly line up and ensure that the target clock frequency can be met.
To do this, the compiler orders primitive operations within each pipeline based on effect and dataflow order.
This ordering is calculated using a reverse depth first search along data and effect dependencies.
A second forward depth first search is then used to minimize delays in reduction cycles.
Based on this ordering, the compiler then inserts pipeline and delay line registers based on lookup tables which map each primitive node to an associated latency. Dependent nodes which have less than a full cycle of delay are kept as combinational operations, with a register only being inserted after the last operation.
This register insertion maximizes the achievable clock frequency for this controller while also minimizing the required initiation interval.

\subsection{Code Generation}
Prior to code generation, the compiler first allocates register names for every \texttt{\small{ArgIn}}, \texttt{\small{ArgOut}}, and \texttt{\small{HostIO}}.
In the final pass over the IR, the code generator then instantiates hardware modules from a library of custom, parameterized RTL templates written in Chisel and infers and generates the logic required to stitch them together.  These templates include state machines that manage communication between the various control structures and primitives in the application, as well as the banked and buffered memory structures and efficient arithmetic operations.  Finally, all generated hardware is wrapped in a target-specific, parameterized Chisel module that arbitrates off-chip accesses from the accelerator with the peripheral devices on the target FPGA.

\section{Evaluation}
\label{spatial-evaluation}

While Spatial's primary goal is to serve as an intermediate hardware abstraction
for compiling DSLs to FPGAs, this level of abstraction also makes it competitive
in terms of programmer productivity with existing HLS tools.
In this section, we first evaluate both of our approaches to automated design tuning,
first with random search with heuristic pruning,
then using the HyperMapper design tuning approach.
Using the generated points from the random search, we then
evaluate the Spatial compiler's area and runtime modeling accuracy.
We then evaluate Spatial and frontend language by comparing programmer productivity
and the performance of generated designs to Xilinx's commercial HLS tool, SDAccel.

To conlude, we also demonstrate Spatial's generality by showing its ability
to target multiple kinds of reconfigurable hardware from the same source code.
showing both portability across FPGAs and on the recently proposed Plasticine CGRA~\cite{plasticine}.

\subsection{Design Tuning with Random Search}

\begin{table}
\centering\footnotesize
\begin{tabular}{lm{3.6cm}m{2cm}}
\toprule
{\bf Benchmark} & {\bf Description} & {\bf Dataset Size} \\ \midrule
dotproduct   & Vector dot product             & 187,200,000\\ \midrule
outerproduct & Vector outer product           & $38,400$ $38,400$ \\ \midrule
gemm         & Tiled matrix multiplication    & $1536 \times 1536$\\ \midrule
tpchq6       & TPC-H Query 6                  & N=18,720,000 \\ \midrule
blackscholes & Black-Scholes-Merton model     & N=9,995,328 \\ \midrule
gda          & Gaussian discriminant analysis & R=360,000 D=96 \\ \midrule
kmeans       & $k$-Means clustering           & \#points=960,000, k=8, dim=384 \\ \midrule
\end{tabular}
\caption{Evaluation benchmarks.}
\label{t:benchmarks}
\end{table}

\subsubsection{Benchmarks}
Table~\ref{t:benchmarks} lists the benchmarks we use in the evaluation of our
design tuning with random search along with the corresponding
input dataset sizes used. \emph{Dotproduct}, \emph{outerprod}, and \emph{gemm} are common
linear algebra kernels. \emph{Tpchq6} is a data analytics application that streams through a collection
of records and performs a reduction on records filtered by a condition. \emph{BlackScholes} is
a financial analytics application that implements Black-Scholes option pricing. \emph{Gda}
and \emph{kmeans}
are commonly used machine learning kernels used for data classification and clustering, respectively.
All benchmarks operate on single-precision floating point numbers,
except in certain cases where the benchmark requires integer or boolean values as inputs.
All benchmarks were written in Spatial by hand but are equivalent to what
could be generated automatically from higher level DSLs.

\subsubsection{Methodology}
Each generated design is synthesized and run on an Altera 28nm Stratix V FPGA
on a Max4 MAIA board at a fabric clock frequency of 150MHz. The MAIA board interfaces with an Intel CPU
via PCIe. The board has 48GB of dedicated off-chip DDR3 DRAM with a peak bandwidth of 76.8GB/s. In practice,
our maximum memory bandwidth is 37.5 GB/s, as our on-chip memory clock is limited to 400MHz.
For this board, we leverage Maxeler's runtime to manage communication and data movement
between the host CPU and the MAIA board.
Execution time is measured starting
from when the FPGA design is started (after input has been copied to FPGA DRAM) and stopped after
the design finishes execution (before output is copied to CPU DRAM). We report execution time as the
average of 20 runs to eliminate noise from design initialization time and off-chip memory latencies.
The FPGA resource utilization
numbers reported are from the post place-and-route report generated by Altera's logic synthesis tools. Logic (LUT) resources are reported in terms of Altera's adaptive logic module (ALM) resources.

\begin{figure*}[!htbp]
\centering
\includegraphics[width=0.95\textwidth]{figs/tradeoff.pdf}
\caption{Results of design space exploration. Horizontal axis shows estimated ALM, DSP, and BRAM usages. Vertical axis shows runtime in cycles, given in log scale (base 10).}
\label{fig:dse}
\end{figure*}

\subsubsection{Pareto-Optimality Analysis}
Figure~\ref{fig:dse} shows the resulting design space scatter plots for all benchmarks in
Table~\ref{t:benchmarks}. A design point is considered
invalid if its resource requirement for at least one type of resource exceeds the maximum
available amount on the target device. Pareto-optimal designs along the dimensions of execution time
and ALM utilization are highlighted for each benchmark through all three resource plots.
We now analyze each benchmark in detail.

\textbf{Dot product} (Figure~\ref{fig:dse} A,B,C) is a memory-bound benchmark. Peak execution
time is reached by balancing tile loads and computation. Inner and outer loop parallelization allows us to
quickly reach close to the input bandwidth. Runtimes of designs with pipelined loops then slowly decrease as parallelization
increases once the dominant stage becomes the dot product reduction tree. In \emph{dotproduct}, designs with pipelines consume fewer resources than those with \emph{Sequential} for the same performance. \emph{Sequential}s require larger tile sizes and more parallelism to match pipelined performance.

\textbf{Outer product} (Figure~\ref{fig:dse} D,E,F) represents both a BRAM and memory bound benchmark. For $2N$ inputs,
the total BRAM requirement is $2N + N^2$ to store the input and output tiles, meaning the BRAM requirement
increases quadratically with increases in input tile size. The highest performing designs for outer product do not use
pipelines to overlap loading and storing of tiles. This is because the overhead due to main memory contention from overlapping
tile loads and stores turns out to be higher than the cost of executing each stage sequentially.

\textbf{GEMM} (Figure~\ref{fig:dse} G,H,I) contains a lot of temporal and spatial locality. From Figure~\ref{fig:dse}(I), Pareto-optimal designs for \emph{gemm}
occupy almost all BRAM resources on the board. Intuitively, this is  because good designs for \emph{gemm}
maximize locality by retaining large, two dimensional chunks of data in on-chip memory.
%However, as seen in Figure~\ref{fig:dse}(G), \emph{gemm} is also limited by the
%number of ALMs due to the large number of floating point operations being done in parallel.

\textbf{TPC-H Q6} (Figure~\ref{fig:dse} J,K,L) exhibits behavior typical of memory intensive applications. Performance reaches a maximum threshold
with increased tile size because of overlapping memory access and compute.

\textbf{BlackScholes} (Figure~\ref{fig:dse} M,N,O) streams through multiple large arrays and performs complex floating point computations
on the input data. Points along the same vertical bar in Figure~\ref{fig:dse}(M) share the same inner loop parallelization
factor. Increasing parallelization improves performance by increasing utilization of the available off-chip memory bandwidth.
Our model suggests that increasing the inner loop parallelization would continue to scale
performance until a parallelization of 16, around which point \emph{blackscholes} would be memory bound. Because there are not enough compute resources are available to implement a parallelization factor of 16, \emph{blackscholes} is ALM bound.

\textbf{GDA} (Figure~\ref{fig:dse} P,Q,R) possesses higher degrees of spatial locality. Because of this, \emph{gda} exhibits compute-bound behavior, where execution time
decreases steadily with increased resource utilization, as seen in Figure~\ref{fig:dse}(P). The critical resource is again BRAM. This is because BRAM usage increases with parallelization due to the creation of
more banks with fewer words per bank, which can cause under-utilization of the capacity of individual BRAMs.

\textbf{K-Means} (Figure~\ref{fig:dse} S,T,U) is bound by the number of ALMs. The critical path in this application is the distance computation done comparing an input point to each centroid.
The number of floating point operations to be done to keep up with main memory bandwidth is therefore proportional to $K \times D$, where $D$ is the number of dimensions in one point.
The performance of \emph{kmeans} is therefore limited by the number of ALMs on the FPGA, as not enough are available to perform all $K \times D$ operations in parallel.
Like GDA, \emph{kmeans} is also limited by BRAMs due to under-utilization of BRAM capacity with increased banking factors.

From our experiments, we observe that capturing parallelism at multiple levels using pipelined controllers enables us to generate
efficient designs. In addition, effective management of on-chip BRAM resources is critical to good designs
as BRAM resources are the limiting factor for performance scaling in most of our benchmarks.

\subsection{Design Tuning Speed}
We compare the speed of our estimation and design space exploration with
Vivado HLS~\cite{vivadohls}, a commercial high-level synthesis tool from Xilinx.
Our evaluation uses the GDA example in Figure~\ref{fig:gda-hls} as input to the high-level synthesis tool, and a GDA design written in Spatial as input to our design space exploration tool. Design parameters
for the high-level synthesis tool are the unrolling factors. We also include a pipeline directive
toggle for each loop in the design. Design tuning speed
is measured by comparing the average estimation speed per point for 250 design points for each tool.
In our experiments, our analysis takes 5 to 29 milliseconds per design depending on the size of the application's intermediate representation. Analysis of GDA also takes 17 milliseconds per design.

\begin{table*}
\centering\footnotesize
\begin{tabular}{lcc}
\toprule
{\bf Our approach}  & {\bf Vivado HLS restricted\textsuperscript{$\dagger$}} & {\bf Vivado HLS full} \\ \midrule
0.017s / design     & 4.75s / design               & 111.06s / design      \\ \midrule
\end{tabular}

\vspace{5pt}
\textsuperscript{$\dagger$}Vivado HLS restricted design space ignores outer loop pipelining
\caption{Average estimation time per design point.}
\label{t:speeds}
\end{table*}

Table~\ref{t:speeds} shows a comparison between estimation speeds from our toolchain and Vivado HLS.
The ``restricted'' column refers to the average time spent per design over points whose outer loop ($L1$, in Figure~\ref{fig:gda-hls})
is not pipelined with a pipeline directive. The ``full'' version refers to all design points where 30 of the 250 points
have a pipeline directive to enable outer loop pipelining. We observe the following:
\begin{itemize}
  \item Our estimation tool is 279$\times$ faster than the ``restricted'' space exploration, and 6533$\times$ faster than the ``full'' space exploration.
  \item Compared to Vivado HLS, our estimation time is not sensitive to design parameter inputs. Estimation time for Vivado HLS increases
    dramatically when the outer loop is pipelined in GDA because the tool completely unrolls all inner loops
    before pipelining the outer loop. This creates a large graph that complicates scheduling. Our approach does not
    suffer from this limitation because we explicitly capture pipelines with parameterized controllers, thereby capturing outer loop pipelining more naturally.
\end{itemize}

\subsection{Design Tuning with Active Learning}
We next perform a preliminary evaluation of HyperMapper for quickly approximating Pareto frontier over two design objectives: design runtime and FPGA logic utilization (LUTs).
For this evaluation, we run HyperMapper with several seeds of initial random sample, with the number of samples $R$ ranging from 1 to 6000 designs, and run 5 active learning iterations of at most 100 samples each. For comparison, the heuristic search proposed in the DHDL work~\cite{dhdl} prunes using simple heuristics and then randomly samples up to 100,000 points. For both approaches, design tuning takes up to 1 -- 2 minutes, varying slightly by benchmark complexity.

Figure~\ref{hvi_samples} shows the hypervolume indicator (HVI) function
for the \emph{BlackScholes} benchmark as a function of the initial
number of random samples.
The HVI gives the area between the estimated Pareto frontier and the space's true Pareto curve, found from exhaustive search.
By increasing
the number of random samples to bootstrap the active learning phase,
we see two orders
of magnitude improvement in HVI. Furthermore, the overall variance goes down very
quickly as the number of random
samples increases. As a result, the autotuner is robust
to randomness and only a handful of random samples are
needed to bootstrap the active learning phase. As shown in Figure~\ref{paretos},
HyperMapper is able to reach a close approximation of
the true Pareto frontier with less than 1500 design points.

On benchmarks like GDA with sparser design spaces, HyperMapper spends much of its time evaluating areas
of the space with invalid designs which cannot fit on the FPGA.
HyperMapper's accuracy for these benchmarks is consequently lower than the heuristic approach.
Consequently, followup work is looking at expanding HyperMapper with a
valid design prediction mechanism~\cite{hypermapper2}.

\begin{figure}
\centering
%%% trim = left, bottom, right, top
\begin{subfigure}[t]{0.45\linewidth}
\includegraphics[width=\linewidth]{5-compiler/figs/hvi_5NumberSummary_median.png}
\subcaption{HyperMapper HVI versus initial random samples ($R$) five number summary.}
\label{hvi_samples}
\end{subfigure}\hspace{15pt}
~
\begin{subfigure}[t]{0.45\linewidth}
\includegraphics[width=\linewidth]{5-compiler/figs/output_pareto_blackscholes.pdf}
\subcaption{Exhaustive and HyperMapper ($R$=1000) generated Pareto curves. }
\label{paretos}
\end{subfigure}

\vspace{-10pt}
\caption{Design space tuning on \emph{BlackScholes}.}
\label{figHVI}
\end{figure}


\subsection{Modeling Accuracy}

\begin{table}
\centering\footnotesize
\begin{tabular}{lcccc}
\toprule

{\bf Benchmark} & {\bf ALMs} & {\bf DSPs} & {\bf BRAM} & {\bf Runtime} \\ \midrule
dotproduct      & 1.7\%      & 0.0\%      & 13.1\%      & 2.8\%  \\ \midrule
outerproduct    & 4.4\%      & 29.7\%     & 12.8\%      & 1.3\%  \\ \midrule
gemm            & 12.7\%     & 11.4\%     & 17.4\%      & 18.4\% \\ \midrule
tpchq6          & 2.3\%      & 0.0\%      & 5.4\%       & 3.1\%  \\ \midrule
blackscholes    & 5.3\%      & 5.3\%      & 7.0\%       & 3.4\%  \\ \midrule
gda             & 5.2\%      & 6.2\%      & 8.4\%       & 6.7\%  \\ \midrule
kmeans          & 2.0\%      & 0.0\%      & 21.9\%      & 7.0\%  \\ \midrule \midrule

Average         & 4.8\%      & 7.5\%     & 12.3\%      & 6.1\%   \\ \bottomrule
\end{tabular}
\caption{Average absolute error for resource usage and runtime.}
\label{t:errors}
\end{table}

We next evaluate the absolute accuracy of our modeling approach. We select five Pareto points generated from our design space exploration for each of our benchmarks. We then generate and synthesize hardware for each design and run it on the FPGA. We compare our area estimates to post place-and-route reports generated by Altera's toolchain. We then run the design on the FPGA and compare the estimated runtime to observed runtime. Note that runtime includes off-chip memory accesses from the FPGA to its DRAM. Table~\ref{t:errors} summarizes the errors averaged across all selected Pareto points for each benchmark.

Our area estimates have an average error of 4.8\% for ALMs, 7.5\% for DSPs, and 12.3\% for BRAMs, while our runtime estimation error averages 6.1\%.
Our highest error occurs in predicting DSPs for \emph{outerprod}, where we over-predict by 29.7\% DSP usage on average.
However, we found that errors above 10\% for DSP usage only occur for designs which use less than 2\% of the total DSPs available on the device.
As our benchmarks are limited by other resources (typically ALMs or BRAM), the relative error for DSPs is more sensitive
to low-level fluctuations and noise. We observe that our DSP estimates preserve absolute ordering of resource utilization. Hence, this error does not
affect the quality of the designs found during design space exploration.%, and improves with increased resource utilization.

Of our estimated metrics, BRAM estimates have the highest average error over all benchmarks. These errors are primarily from block RAM duplication done by the placement and routing tool. In designing our models, we found that BRAM duplication is inherently noisy, as more complex machine learning models failed to achieve better estimates than a simple linear fit. Our linear model provides a rough estimate of design complexity and routing requirements, but it does not provide a complete picture for when and how often the synthesis tool will decide to duplicate BRAMs. However, like DSPs, we find that our BRAM estimates track actual usage and preserve ordering across designs, making it usable for design space exploration and relative design comparisons.

\emph{GEMM} has the highest overall error of any benchmark. We found that this is due to low-level hardware optimizations like floating point multiply-add fusion, fusion of floating point reduction trees, and BRAM coalescing that Maxeler's compiler performs automatically and that we use heuristics to predict. Since we do not have explicit control over these optimizations, it is possible to mispredict when they will occur. The \emph{gemm} benchmark is exceptionally sensitive to these errors. However, as with the other errors, we found that this error does not detract from the model's ability to guide design space exploration as long as the possibility of this error is accounted for.


\subsection{FPGA Performance and Productivity}

\input{5-compiler/figs/eval_sdaccel}

We next evaluate the FPGA performance and productivity benefits of Spatial against SDAccel, a commercial C-based programming tool from Xilinx for creating high-performance accelerator designs. We use SDAccel in our study as it has similar performance and productivity goals as Spatial and performs several optimizations related to loop pipelining, unrolling, and memory partitioning~\cite{sdaccel}. Baseline implementations of the benchmarks in Table~\ref{t:hls_comp} have been either obtained from a public SDAccel benchmark suite from Xilinx~\cite{sdaccelBench}, or written by hand. Each baseline has then been manually tuned by using appropriate HLS pragmas~\cite{hlsPragmaRef} to pick loop pipelining, unrolling, and array banking factors, and to enable dataflow optimizations. Design points for Spatial are chosen using the DSE flow described in Section~\ref{dse}.

We measure productivity by comparing number of lines of source code used to describe the FPGA kernel, excluding host code. We measure performance by comparing runtimes and FPGA resources utilized for each benchmark on a Xilinx Ultrascale+ VU9P board with a fabric clock of 125 MHz, hosted on an Amazon EC2 F1 instance. We generate FPGA bitstreams targeting the VU9P architecture for each benchmark using both Spatial and SDAccel, and obtain resource utilization data from the post place-and-route reports. We then run and verify both designs on the FPGA and measure the execution times on the board. CPU setup code and data transfer time between CPU and FPGA is excluded from runtime measurements for both tools.

Table~\ref{t:hls_comp} shows the input dataset sizes and the full comparison between lines of source code, resource utilization, and runtime of the benchmarks implemented in SDAccel and Spatial.
In terms of productivity, language constructs in Spatial like \texttt{\small{load}} and \texttt{\small{store}} for transferring dense sparse data from DRAM reduces code bloat and increases readability.
Furthermore, by implicitly
inferring parameters such as parallelization factors and loop initiation intervals, Spatial code is largely free of annotations and pragmas.

Spatial achieves speedups over SDAccel of $1.63\times$ and $1.33\times$ respectively on \emph{BlackScholes} and \emph{TPC-H Q6}. Both benchmarks
stream data from DRAM through a deeply pipelined datapath which is amenable to FPGA acceleration. Dataflow support in SDAccel using the DATAFLOW pragma~\cite{dataflowRef} and streaming support in Spatial allows both tools to efficiently accelerate such workloads. In \emph{K-Means}, coarse-grained pipelining support allows Spatial to achieve roughly the same performance as SDAccel using $1.5\times$ fewer BRAMs.
Specialized DRAM scatter/gather support enables Spatial to achieve a $3.48\times$ speedup on \emph{PageRank}.

We see speedups of $8.48\times$, $1.37\times$, and $14.15\times$ for compute-heavy workloads \emph{GDA}, \emph{GEMM}, and \emph{SW}, respectively. The baseline for \emph{SW} is implemented by Xilinx as a systolic array, while the Spatial implementation uses nested controllers. \emph{GEMM} and \emph{GDA} contain opportunities for coarse-grained pipelining that are exploited within Spatial.
GDA, for example, contains an outer product operation, during which the data in the same buffer is repeatedly accessed and reused. While this operation can be pipelined with a preceding loop producing the array, SDAccel's DATAFLOW pragma does not support such access patterns that involve reuse. As a result, SDAccel requires larger array partitioning and loop unrolling factors to offset the performance impact, at the expense of consuming more FPGA BRAM.
In addition, nested controllers in \emph{GEMM}
can be parallelized and pipelined independently in Spatial, while SDAccel automatically unrolls all inner loops if an outer loop is parallelized. Spatial can therefore explore
design points that cannot be easily expressed in SDAccel. Finally, as the Spatial compiler performs analyses on a parameterized IR, the compiler can reason about larger parallelization factors without expanding the IR graph.
SDAccel unrolls the graph as a preprocessing step, hence creating larger graphs when unrolling and array partitioning factors are large.
This has a significant impact on the compiler's memory footprint and compilation times, making better designs difficult or impossible to find.

Spatial provides a productive platform to program FPGAs, with a 42\% reduction in lines of code compared to SDAccel averaged across all benchmarks. On the studied benchmarks, Spatial achieves a geometric mean speedup of $2.9\times$ compared to an industrial HLS tool.


\subsection{Spatial FPGA Portability}

\begin{table}
\centering
\fontsize{7}{9}\selectfont
\begin{tabular}{l d{2.1} d{2.1} d{2.1} d{2.1} d{2.1} d{2.1}}
   \bf{FPGA}      & \mc{ZC706}  & \multicolumn{4}{c}{\bf VU9P}                                       & \mc{Total}     \\
   \bf{Design}    & \mc{Tuned}  & \multicolumn{2}{c}{\bf Ported}   & \multicolumn{2}{c}{\bf Tuned}    &               \\ \toprule

                  & \mc{Time}   & \mc{Time}  & \mc{$\times$}       & \mc{Time}  & \mc{$\times$}       & \mc{$\times$} \\ \midrule
   \ml{BS}        & 89.0        & 35.6       & 2.5                 & 3.8        & 9.4                 & 23.4          \\ \midrule
   \ml{GDA}       &  8.4        & 3.4        & 2.5                 & 1.3        & 2.6                 & 6.5           \\ \midrule
   \ml{GEMM}      & 2226.5      & 1832.6     & 1.2                 & 878.5      & 2.1                 & 2.5           \\ \midrule
   \ml{KMeans}    & 358.4       & 143.4      & 2.5                 & 53.3       & 2.7                 & 6.7           \\ \midrule
   \ml{PageRank}  & 1299.5      & 1003.3     & 1.3                 & 587.4      & 1.7                 & 2.2           \\ \midrule
   \ml{SW$^\dag$} & 1.3         &  0.5       & 2.5                 & 0.5        & 1.0                 & 2.5           \\ \midrule
   \ml{TQ6}       & 69.4        & 15.0       & 4.6                 & 14.0       & 1.1                 & 5.0           \\ \bottomrule

   \multicolumn{7}{l}{\vspace{10pt}\footnotesize{ $^\dag$SW with 160 base pairs, the largest to fit on the ZC706.}}
\end{tabular}
\caption{Runtimes (ms) of tuned designs on ZC706, followed by runtimes and speedup~($\times$) of directly porting these designs to the VU9P, then runtimes and successive speedup over ported designs when tuned for the VU9P. The \emph{Total} column shows the cumulative speedup.}
\label{fig:zynq_comp}
\end{table}

Next, we demonstrate the portability of Spatial code by targeting two different FPGA architectures; (1) the Zynq ZC706 SoC board, and (2) The Virtex Ultrascale+ VU9P on the Amazon EC2 F1.
Designs on the VU9P use a single DRAM channel with a peak bandwidth of 19.2 GB/s. The ZC706 is much smaller than the VU9P in terms of FPGA resource and has a smaller DRAM bandwidth of 4.26 GB/s.
We target both the ZC706 and VU9P from the same Spatial code for all benchmarks listed in Table~\ref{t:hls_comp}. Benchmarks are tuned for each target using
target-specific models with automated DSE. Clock frequency is fixed at 125 MHz for both FPGAs.

Table~\ref{fig:zynq_comp} shows the speedups achieved on the VU9P over the ZC706. The results show that not only can the same Spatial source code be ported
to architectures with different capabilities, the application can also be automatically tuned to better take advantage of resources in each target.
Compute-bound benchmarks \emph{BlackScholes}, \emph{GDA}, \emph{GEMM}, \emph{K-Means} achieve speedups of up to $23\times$ on the VU9P over the ZC706. Porting these designs to the VU9P alone has a $1.2\times$ to $2.5\times$ due to increased main memory bandwidth, but a majority of the benefit of the larger FPGA comes from tuning the parallelization factors to use more resources.
While \emph{SW} is also compute bound, the size of the dataset was limited by the smaller FPGA. In this case, the larger capacity of the VU9P does not improve runtime, but instead allows handling of larger datasets.

Memory-bound benchmark \emph{TPC-H Q6} benefits from the higher DRAM bandwidth available on the VU9P. Porting this benchmark immediately gives a $4.6\times$ runtime improvement from the larger main memory bandwidth, while further parallelizing controllers to create more parallel address streams to DRAM helps the application make better use of this bandwidth. \emph{PageRank} is also bandwidth-bound, but the primary benefit on the VU9P comes from specializing the memory controller to maximize utilized bandwidth for sparse accesses.


% \begin{table}
% \caption{Speedup of VU9P over ZC706.}
% \label{fig:zynq_comp}

% \fontsize{8}{10}\selectfont
% \begin{tabular}{cccccc}
% \bf{GDA}     & \bf{GEMM}    & \bf{K-Means} & \bf{PageRank} & \bf{SW}      & \bf{TQ6} \\ \hline
% 2.54$\times$ & 2.53$\times$ & 1.84$\times$ & 2.21$\times$  & 1.74$\times$ & 4.97$\times$  \\ \hline
% \end{tabular}
% \end{table}

\subsection{Spatial Generality Beyond FPGAs}
\label{plasticine-evaluation}

\begin{table}
\centering
\fontsize{7}{7}\selectfont
  \begin{tabular}{m{0.5cm} d{2.1} d{2.1} d{2.1} d{2.1} d{2.1} d{2.1} r }
  \toprule
                 & \multicolumn{2}{c}{\bf Avg DRAM } & \multicolumn{3}{c}{\bf Resource }        & \mc{}     & \mc{} \\
                 & \multicolumn{2}{c}{\bf BW (\%)}   & \multicolumn{3}{c}{\bf Utilization (\%)} & \mc{Time} & \mc{$\times$} \\
   \bf{App}     & \mc{Load}    & \mc{Store} & \mc{PCU}  & \mc{PMU}  & \mc{AG}   & \mc{(ms)} & \mc{} \\ \midrule
   \ml{BS}       & 77.4        & 12.9       & \mb{\hspace{1pt}73.4} & 10.9      & 20.6      & 2.33      & 1.6   \\
   \ml{GDA}      & 24.0        & 0.2        & \mb{\hspace{1pt}95.3} & 73.4      & 38.2      & 0.13      & 9.8   \\
   \ml{GEMM}     & 20.5        & 2.1        & \mb{\hspace{1pt}96.8} & 64.1      & 11.7      & 15.98     & 55.0  \\
   \ml{KMeans}   & 8.0         & 0.4        & \mb{\hspace{1pt}89.1} & 57.8      & 17.6      & 8.39      & 6.3   \\
   \ml{TQ6}      & \mb{\hspace{2pt}97.2}   & 0.0        & 29.7      & 37.5      & \mb{70.6} & 8.60      & 1.6   \\

\bottomrule
\end{tabular}
\caption{Plasticine DRAM bandwidth, resource utilization, runtime, and speedup ($\times$) over a Xilinx
VU9P FPGA. Both Plasticine and FPGA implementations were generated from the same Spatial source code.}
\label{table:plasticine_eval}
\end{table}

Finally, we demonstrate the portability of Spatial beyond
FPGA architectures and its generality across reconfigurable architectures
by including an evaluation of the compiler to map
the Spatial IR to target the Plasticine CGRA~\cite{plasticine,plasticine2}. Plasticine is a
two-dimensional array of compute (PCUs) and memory
(PMUs) tiles with a statically configurable interconnect
and address generators (AG) at the periphery to perform
DRAM accesses. The Plasticine architecture is a significant departure
from an FPGA, with more constraints on memory banking and computation, including
fixed size, pipelined SIMD lanes. More details about the compiler extensions
required to target Plasticine can be found in the original work
on the Plasticine CGRA~\cite{plasticine}.

We simulate Plasticine with  a $16 \times 8$ array of 64 compute and 64 memory tiles, with a 1 GHz clock and a main memory with a DDR3-1600 channel with 12.8 GB/s peak bandwidth.
Table~\ref{table:plasticine_eval} shows the DRAM bandwidth, resource utilization, runtime, and speedup of the Plasticine CGRA over the VU9P for a subset of benchmarks.

Streaming, bandwidth-bound applications like \emph{TPC-H Q6} efficiently exploit about 97\% of the available DRAM bandwidth.
Compute-bound applications \emph{GDA}, \emph{GEMM}, and \emph{K-Means} use around 90\% of Plasticine's compute tiles.
Plasticine's higher on-chip bandwidth also allows these applications to better utilize the compute resources, giving these applications speedups of $9.9\times$, $55.0\times$, and $6.3\times$.
Similarly, the deep compute pipeline in \emph{BlackScholes} occupies 73.4\% of compute resources after being split across multiple tiles,
giving a speedup of $1.6\times$.

\section{Conclusion}

In this chapter, we discussed the Spatial compiler and the various optimizations
it performs, including control scheduling, memory banking and buffering,
design modeling and hardware parameter tuning, and pipeline retiming.
We showed that our modeling approach has an average area estimation
error of 4.8\% and average runtime estimation error of 6.1\% over all the benchmarks.
We performed a detailed study for each benchmark on the space of designs described by tile sizes, parallelism
factors, and coarse-grained pipelining and measure their effects on the utilization of different types of FPGA
resources. We show that the design tuning within our compiler can run 279 to 6533 times
faster than a commercial high-level synthesis tool. We also showed the potential of
more advanced search strategies like active learning with a preliminary evaluation
of HyperMapper integrated with Spatial.

Most importantly, we have demonstrated that Spatial can target a range of reconfigurable
architectures, including FPGAs and the Plasticine CGRA,
from a single source. Spatial can achieve average speedups
of $2.9\times$ over SDAccel while also being more productive for power users,
with an average of 42\% fewer lines of code when defining the same applications.

