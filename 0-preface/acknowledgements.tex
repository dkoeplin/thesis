Like so many systems projects, the work in this thesis would not have been
possible (or at least would have taken much, much longer) without the help of
a great number of people. First and foremost, my advisor Kunle Olukotun.

Kunle's role in my PhD career and in this project can perhaps be summarized by
our
first interaction together. After I described my research interests in machine
learning and reconfigurable computing, Kunle suggested that I review recent
work in the area to see if we might think of a way to do something better.
``Or different,'' I suggested, to which he quickly responded ``let's stick
with better.''  Ever since then, Kunle been the guiding force helping me and this
project move forward and stay out of the weeds.

Key components of this work were the result of long discussions and even longer
working hours by various members of the Pervasive Parallelism Lab, including
Raghu Prabhakar, Matt Feldman, Yaqi Zhang, Tian Zhao, Stefan Hadjis, and Matt
Vilim. For all my work in the compiler's guts, my colleagues have been largely
responsible for Spatial's code generator backends, without which the
compiler really wouldn't be much of a compiler at all.

I received a lot of excellent early insights into this project from my senior
colleagues in the PPL group, including Arvind Sujeeth, Kevin Brown, HyoukJoong
Lee, Chris De Sa, and Kevin Conley. Chief among these insights was the use of
Scala and parallel patterns, both of which I have since grown to (mostly) love.

I have also had extremely helpful discussions on the design of Spatial
with a multitude of other people, including (but certainly not limited to)
Christos Kozyrakis, Kayvon Fatahalian, Mark Horowitz,
Luigi Nardi, Ardavan Pedram, Ruben Fiszel, Nathan Zhang, Alex Rucker, and
numerous other members of the DAWN research group too numerous to list.

Lastly, I should thank all of my coworkers at SambaNova Systems. Our time
together so far has been wonderful and I am greatly looking forward to what
the future will have to offer us (and, of course, vice versa).
