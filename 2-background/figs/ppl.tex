\begin{figure*}
\centering

%%%%%%%%% ----- Map ------- %%%%%
\newsavebox{\Map}
\begin{lrbox}{\Map}
\begin{lstlisting}[language=PPLTable]
Map(d+){m}: V$_D$
\end{lstlisting}
\end{lrbox}

\newsavebox{\MapHLL}
\begin{lrbox}{\MapHLL}
\begin{lstlisting}[language=PPLTable]
// Vector of size s multiplied by 2
vector * 2
// Addition of two size s vectors
vectorA + vectorB
\end{lstlisting}
\end{lrbox}

\newsavebox{\MapPPL}
\begin{lrbox}{\MapPPL}
\begin{lstlisting}[language=PPLTable]
map(s){i => vector(i) * 2 }

map(s){i => vectorA(i) + vectorB(i) }
\end{lstlisting}
\end{lrbox}

%%%%%%%%% ----- MultiFold ------- %%%%%
\newsavebox{\MultiFold}
\begin{lrbox}{\MultiFold}
\begin{lstlisting}[language=PPLTable]
MultiFold(d+)(r)(z){f}{c}: V$_R$
\end{lstlisting}
\end{lrbox}

\newsavebox{\MultiFoldHLL}
\begin{lrbox}{\MultiFoldHLL}
\begin{lstlisting}[language=PPLTable]
// Reduction of a vector of s elements
vector.product


// Rum summation in an s $\times$ t matrix
x.mapRows{row => row.sum }


\end{lstlisting}
\end{lrbox}

\newsavebox{\MultiFoldPPL}
\begin{lrbox}{\MultiFoldPPL}
\begin{lstlisting}[language=PPLTable]
multiFold(s)(1)(1){ i =>
  (0, acc => acc + x(i))
}{ (a,b) => a + b }

multiFold(s,t)(r)(zeros(s)){ (i,j) =>
  (i, acc => acc + x(i,j) )
}{ (a,b) => map(s){ i => a(i) + b(i) } }
\end{lstlisting}
\end{lrbox}

%%%%%%%%% ----- FlatMap ------- %%%%%
\newsavebox{\FlatMap}
\begin{lrbox}{\FlatMap}
\begin{lstlisting}[language=PPLTable]
FlatMap(d){n}: V$_1$
\end{lstlisting}
\end{lrbox}

\newsavebox{\FlatMapHLL}
\begin{lrbox}{\FlatMapHLL}
\begin{lstlisting}[language=PPLTable]
// Filter positives from s elements
SELECT * FROM vector WHERE elem >= 0
\end{lstlisting}
\end{lrbox}

\newsavebox{\FlatMapPPL}
\begin{lrbox}{\FlatMapPPL}
\begin{lstlisting}[language=PPLTable]
flatMap(s){ i =>
  if (x(i) > 0) [x(i)] else [] }
\end{lstlisting}
\end{lrbox}

%%%%%%%%% ----- GroupByFold ------- %%%%%
\newsavebox{\GroupByFold}
\begin{lrbox}{\GroupByFold}
\begin{lstlisting}[language=PPLTable]
GroupByFold(d)(z){g}{c}: (K,V)$_1$
\end{lstlisting}
\end{lrbox}

\newsavebox{\GroupByFoldHLL}
\begin{lrbox}{\GroupByFoldHLL}
\begin{lstlisting}[language=PPLTable]
// Histogram with bin width 10
x.groupByFold(0){ e =>
  (e/10, 1)
}{ (a,b) => a + b }
\end{lstlisting}
\end{lrbox}

\newsavebox{\GroupByFoldPPL}
\begin{lrbox}{\GroupByFoldPPL}
\begin{lstlisting}[language=PPLTable]
groupByFold(s)(0){ i =>
  (x(i)/10, acc => acc + 1)
}{ (a,b) => a + b }
\end{lstlisting}
\end{lrbox}

\fontsize{10}{11}
\selectfont
\begin{tabular}{lll}
{\begin{tabular}{l}
\multicolumn{1}{l}{\bf{(a) Multi-Dimensional}} \\ \midrule
%%{\bf Parallel Pattern Definition} & {\bf High Level Language Examples} & {\bf PPL Example} \\ \hline
\multirow{1}{*}{\usebox{\Map}} \\
\vspace{-7pt} \\
Mapping over a $D$-dimensional domain. \\
\vspace{-8pt} \\
~~~\argg{d}: $\mathbb{Z}^*_D$ $D$-dimensional domain. \\
~~~\argg{m}: $\mathbb{Z}^*_D \rightarrow V$ Value function. \\
\vspace{-9pt} \\

\multirow{1}{*}{\usebox{\MultiFold}} \\
\vspace{-7pt} \\
Combination of $R$-dimensional blocks over \\
a $D$-dimensional iteration domain. \\
\vspace{-8pt} \\
~~~\argg{d}: $D$-dimensional domain ($\mathbb{Z}^*$+). \\
~~~\argg{r}: Output range ($\mathbb{Z}^*_R \rightarrow V$). \\
~~~\argg{z}: Initial accumulator ($\mathbb{V}_R$). \\
~~~\argg{f}: (Location, Value) function. ($\mathbb{Z}^*_R \rightarrow (\mathbb{Z}^*_R, V_R \rightarrow V_R)$) \\
~~~\argg{c}: $(V_R, V_R) \rightarrow V_R$ Combine function. \\
\vspace{-9pt} \\
\end{tabular}}
& {}
& {
\begin{tabular}{l}
\multicolumn{1}{l}{\bf{(b) One Dimensional}} \\ \midrule
\multirow{1}{*}{\usebox{\FlatMap}} \\
\vspace{-7pt} \\
Concatenation of 1-dimensional blocks \\
over a 1-dimensional iteration domain. \\
\vspace{-8pt} \\
~~~\argg{d}: 1-dimensional domain ($\mathbb{Z}^*_0$) \\
~~~\argg{n}: Multi-value function ($V_D$) \\
\vspace{-9pt} \\

\multirow{1}{*}{\usebox{\GroupByFold}} \\
\vspace{-7pt} \\
Concatenation of 1-dimensional blocks \\
over a 1-dimensional iteration domain. \\
\vspace{-8pt} \\
~~~\argg{d}: 1-dimensional domain ($\mathbb{Z}^*_0$) \\
~~~\argg{z}:  \\
~~~\argg{g}:  \\
~~~\argg{c}:  \\
\vspace{-9pt} \\

\end{tabular}
}
\end{tabular}

%{\usebox{\MapHLL}} &
%{\usebox{\MapPPL}} \\ \arrayrulecolor[gray]{0.8}\hline

%{\usebox{\MultiFold}} & {\usebox{\MultiFoldHLL}} & {\usebox{\MultiFoldPPL}} \\ \arrayrulecolor[gray]{0.8}\hline

%{textbf{\texttt{One-dimensional}}} & {} & {} \\
%{\usebox{\FlatMap}} & {\usebox{\FlatMapHLL}} & {\usebox{\FlatMapPPL}} \\ \arrayrulecolor[gray]{0.8}\hline
%{\usebox{\GroupByFold}} & {\usebox{\GroupByFoldHLL}} & {\usebox{\GroupByFoldPPL}} \\
%
% {\begin{tabular*}{0.95\textwidth}{llll}
% {\bf User-defined Values} & & & \\ \hline
% {\begin{lstlisting}[language=PPLTable]
% d : Integer$_D$
% \end{lstlisting}} &
%
% {\begin{lstlisting}[numbers=none,mathescape=true]
% input domain
% \end{lstlisting}} &
%
% {\begin{lstlisting}[numbers=none,mathescape=true]
% m : Index$_D$ => V
% \end{lstlisting}} &
%
% {\begin{lstlisting}[numbers=none,mathescape=true]
% value function
% \end{lstlisting}} \\
%
% {\begin{lstlisting}[numbers=none,mathescape=true]
% r : Integer$_R$
% \end{lstlisting}} &
%
% {\begin{lstlisting}[numbers=none,mathescape=true]
% output range
% \end{lstlisting}} &
%
% {\begin{lstlisting}[numbers=none,mathescape=true]
% n : Index => V$_1$
% \end{lstlisting}} &
%
% {\begin{lstlisting}[numbers=none,mathescape=true]
% multi-value function
% \end{lstlisting}} \\
%
% {\begin{lstlisting}[numbers=none,mathescape=true]
% z : V$_R$
% \end{lstlisting}} &
%
% {\begin{lstlisting}[numbers=none,mathescape=true]
% init accumulator
% \end{lstlisting}} &
%
% {\begin{lstlisting}[numbers=none,mathescape=true]
% f : Index$_D$ => (Index$_R$, V$_R$ => V$_R$)
% \end{lstlisting}} &
%
% {\begin{lstlisting}[numbers=none,mathescape=true]
% (location, value) function
% \end{lstlisting}} \\
%
% {\begin{lstlisting}[numbers=none,mathescape=true]
% c : (V$_R$,V$_R$) => V$_R$
% \end{lstlisting}} &
%
% {\begin{lstlisting}[numbers=none,mathescape=true]
% combine accumulator
% \end{lstlisting}} \hspace{51pt} &
%
% {\begin{lstlisting}[numbers=none,mathescape=true]
% g : Index => (K, V => V)$_1$
% \end{lstlisting}} &
%
% {\begin{lstlisting}[numbers=none,mathescape=true]
% (key, value) function
% \end{lstlisting}} \\
% \noalign{\hrule height 1.5pt}
%\end{tabular}

\caption{\label{fig:ppl-syntax}Definitions and usage examples of supported parallel patterns.}
\end{figure*}
