\begin{figure*}
\centering

%%%%%%%%% ----- Map ------- %%%%%
\newsavebox{\Map}
\begin{lrbox}{\Map}
\begin{lstlisting}[language=PPLTable]
Map(d){m}: V$_D$
\end{lstlisting}
\end{lrbox}

%%%%%%%%% ----- MultiFold ------- %%%%%
\newsavebox{\MultiFold}
\begin{lrbox}{\MultiFold}
\begin{lstlisting}[language=PPLTable]
MultiFold(d)(r)(z){f}{c}: V$_R$
\end{lstlisting}
\end{lrbox}

%%%%%%%%% ----- FlatMap ------- %%%%%
\newsavebox{\FlatMap}
\begin{lrbox}{\FlatMap}
\begin{lstlisting}[language=PPLTable]
FlatMap(d){n}: V$_1$
\end{lstlisting}
\end{lrbox}



%%%%%%%%% ----- GroupByFold ------- %%%%%
\newsavebox{\GroupByFold}
\begin{lrbox}{\GroupByFold}
\begin{lstlisting}[language=PPLTable]
GroupByFold(d)(z){g}{c}: (K,V)$_1$
\end{lstlisting}
\end{lrbox}


\fontsize{10}{11}
\selectfont
\begin{tabular}{l}
\multicolumn{1}{l}{\bf{(a) Multi-Dimensional Patterns}} \\ \midrule
\multirow{1}{*}{\usebox{\Map}} \\
\vspace{-7pt} \\
Mapping over a $D$-dimensional domain. \\
\vspace{-8pt} \\
{
\begin{tabular}{lll}
\argg{d}: & \hspace{-7pt}$D$-dimensional iteration domain. & $\mathbb{Z}^D$ \\
\argg{m}: & \hspace{-7pt}Value function.   & $\mathbb{Z}_D\rightarrow V$ \\
\end{tabular}
}
\\
\\
\multirow{1}{*}{\usebox{\MultiFold}} \\
\vspace{-7pt} \\
Combination of $R$-dimensional tensors over a $D$-dimensional iteration domain. \\
\vspace{-8pt} \\
{
\begin{tabular}{lll}
\argg{d}: & \hspace{-7pt}$D$-dimensional iteration domain. & $\mathbb{Z}^D$ \\
\argg{r}: & \hspace{-7pt}Output index range.         & $\mathbb{Z}^R$ \\
\argg{z}: & \hspace{-7pt}Initial accumulator         & $V^R$ \\
\argg{f}: & \hspace{-7pt}(Offset, Value) function.   & $\mathbb{Z}_D \rightarrow (\mathbb{Z}_R, V^R \rightarrow V^R)$ \\
\argg{c}: & \hspace{-7pt}Combine function.           & $(V^R, V^R) \rightarrow V^R$ \\
\end{tabular}
}
\\
\\ \\
\multicolumn{1}{l}{\bf{(b) One Dimensional Patterns}} \\ \midrule
\multirow{1}{*}{\usebox{\FlatMap}} \\
\vspace{-7pt} \\
Concatenation of 1-dimensional arrays over a 1-dimensional iteration domain. \\
\vspace{-8pt} \\
{
\begin{tabular}{lll}
\argg{d}: & \hspace{-7pt}1-dimensional iteration domain. & $\mathbb{Z}^1$ \\
\argg{n}: & \hspace{-7pt}Multi-value function. & $\mathbb{Z} \rightarrow V^1$ \\
\end{tabular}
}
\\
\\
\multirow{1}{*}{\usebox{\GroupByFold}} \\
\vspace{-7pt} \\
Associative reduction of values over a 1-dimensional domain based on paired key values. \\
\vspace{-8pt} \\
{
\begin{tabular}{lll}
\argg{d}: & \hspace{-7pt}1-dimensional iteration domain. & $\mathbb{Z}^1$ \\
\argg{z}: & \hspace{-7pt}Identity value. & $V$ \\
\argg{g}: & \hspace{-7pt}(Key,Value) function. & $\mathbb{Z} \rightarrow (K,V)$ \\
\argg{c}: & \hspace{-7pt}Combination function. & $(V,V) \rightarrow V$ \\
\end{tabular}
} \\
\end{tabular}

%{\usebox{\MapHLL}} &
%{\usebox{\MapPPL}} \\ \arrayrulecolor[gray]{0.8}\hline

%{\usebox{\MultiFold}} & {\usebox{\MultiFoldHLL}} & {\usebox{\MultiFoldPPL}} \\ \arrayrulecolor[gray]{0.8}\hline

%{textbf{\texttt{One-dimensional}}} & {} & {} \\
%{\usebox{\FlatMap}} & {\usebox{\FlatMapHLL}} & {\usebox{\FlatMapPPL}} \\ \arrayrulecolor[gray]{0.8}\hline
%{\usebox{\GroupByFold}} & {\usebox{\GroupByFoldHLL}} & {\usebox{\GroupByFoldPPL}} \\
%
% {\begin{tabular*}{0.95\textwidth}{llll}
% {\bf User-defined Values} & & & \\ \hline
% {\begin{lstlisting}[language=PPLTable]
% d : Integer$_D$
% \end{lstlisting}} &
%
% {\begin{lstlisting}[numbers=none,mathescape=true]
% input domain
% \end{lstlisting}} &
%
% {\begin{lstlisting}[numbers=none,mathescape=true]
% m : Index$_D$ => V
% \end{lstlisting}} &
%
% {\begin{lstlisting}[numbers=none,mathescape=true]
% value function
% \end{lstlisting}} \\
%
% {\begin{lstlisting}[numbers=none,mathescape=true]
% r : Integer$_R$
% \end{lstlisting}} &
%
% {\begin{lstlisting}[numbers=none,mathescape=true]
% output range
% \end{lstlisting}} &
%
% {\begin{lstlisting}[numbers=none,mathescape=true]
% n : Index => V$_1$
% \end{lstlisting}} &
%
% {\begin{lstlisting}[numbers=none,mathescape=true]
% multi-value function
% \end{lstlisting}} \\
%
% {\begin{lstlisting}[numbers=none,mathescape=true]
% z : V$_R$
% \end{lstlisting}} &
%
% {\begin{lstlisting}[numbers=none,mathescape=true]
% init accumulator
% \end{lstlisting}} &
%
% {\begin{lstlisting}[numbers=none,mathescape=true]
% f : Index$_D$ => (Index$_R$, V$_R$ => V$_R$)
% \end{lstlisting}} &
%
% {\begin{lstlisting}[numbers=none,mathescape=true]
% (location, value) function
% \end{lstlisting}} \\
%
% {\begin{lstlisting}[numbers=none,mathescape=true]
% c : (V$_R$,V$_R$) => V$_R$
% \end{lstlisting}} &
%
% {\begin{lstlisting}[numbers=none,mathescape=true]
% combine accumulator
% \end{lstlisting}} \hspace{51pt} &
%
% {\begin{lstlisting}[numbers=none,mathescape=true]
% g : Index => (K, V => V)$_1$
% \end{lstlisting}} &
%
% {\begin{lstlisting}[numbers=none,mathescape=true]
% (key, value) function
% \end{lstlisting}} \\
% \noalign{\hrule height 1.5pt}
%\end{tabular}

\caption{\label{fig:ppl-syntax}Definition of the patterns in the parallel pattern language (PPL). $V^D$ denotes a tensor with $D$ dimensions and elements of type $V$, while $V_D$ denotes a tuple of $D$ elements of type $V$.}
\end{figure*}
