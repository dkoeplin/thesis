\chapter{Introduction}
\label{intro}

%\gist{spatial architectures are important}
Recent trends in technology scaling, the availability of large amounts of data, and novel algorithmic breakthroughs
have spurred accelerator architecture research. Reconfigurable architectures like field-programmable gate arrays (FPGAs) and coarse-grain reconfigurable  architectures (CGRAs)
have received renewed interest from academic researchers and industry practitioners alike, primarily due to their potential performance and energy efficiency benefits over conventional CPUs. 
FPGAs are now being used to accelerate web search
in datacenters at Microsoft and Baidu~\cite{catapult, baidu},
Amazon now offers FPGA instances as part of AWS~\cite{awsf1}, 
and Intel has announced products like in-package Xeon-FPGA systems~\cite{harp}
and FPGA-accelerated storage systems~\cite{nand_flash}.
Similarly, several recent research prototypes~\cite{dyser, ti, scaledeep, scnn, plasticine}
and startups~\cite{wavecomp, nervana} have explored various
kinds of CGRAs at different granularities. 
Growing use of such reconfigurable architectures has made them more available to programmers now than ever before.


Reconfigurable devices are able to accelerate applications, in part, by exploiting multiple levels of nested parallelism and data locality with custom data pipelines and memory hierarchies.
Unfortunately, the same features that make reconfigurable architectures efficient also make them much more complex to program. An accelerator design must account for the timing between pipelined signals and
the physically limited compute and memory resources available on the target device. It must also manage partitioning of data between local scratchpads and off-chip memory to achieve good data locality. 
The combination of these complexities leads to intractable accelerator design spaces~\cite{cascaval}.


These challenges have caused programmability to be a key limiting factor to widespread adoption of CGRAs and FPGAs~\cite{fpgaMasses,DeSutter2013}. 
The space of CGRA programmability is fragmented with incompatible, architecture-specific programming models.
The current state of the art in programming FPGAs involves using a combination of vendor-supplied IP blocks, hand-tuned hardware modules written using either low-level RTL or high-level synthesis tools, and architecture-specific glue logic to communicate with off-chip components such as DRAM. 
Hardware description languages (HDLs) like Verilog and VHDL are designed for explicit specification of hardware, 
placing the burden on the user to solve the complexities of implementing their algorithm in hardware.


%\ardavan{this [area] is optimization so I don't think it belongs here}
%\david{I'm not so sure about that. Limited area resources are a fundamental challenge with targeting an FPGA}
%\ardavan{now you jumped from hard to program to complex design space without a transition.}
%\david{complex design space is part of what makes them hard to program - but this could be clearer}
%This design complexity creates a persistent need for better tools to program reconfigurable devices.
%\todo{For this reason, HDLs are considered prohibitively complicated by most software programmers interested in
%accelerating their applications.}

High-level synthesis (HLS) tools like SDAccel~\cite{sdaccel}, Vivado HLS~\cite{vivadohls}, and Intel's OpenCL SDK~\cite{opencl_sdk} raise the level of abstraction compared to HDLs significantly. 
For example, HLS tools allow programmers to write accelerator designs in terms of untimed, nested loops and offer library functions for common operations like data transfer between a CPU host and the FPGA.  
However, existing commercial HLS tools have all been built on top of software languages like C, OpenCL, and Matlab. 
These software languages have been built to target instruction-based processors like CPUs and GPUs. 
Consequently, although existing HLS tools raise the level of abstraction for targeting reconfigurable architectures, they do so with an ad-hoc, often underspecified mix of software and hardware abstractions.  
For instance, while SDAccel can convert nested loops into hardware state machines, the language has no notion of the architecture's memory hierarchy and cannot pipeline loops at arbitrary nesting levels~\cite{vivado_userguide}.
Programmers must keep in mind that, despite the software programming abstractions, they must employ hardware, not software, optimization techniques.
This makes it challenging to write HLS code which produces fully optimized designs~\cite{nane2016survey}.
%As a result, it can be quite challenging to write HLS programs which produce fully optimized designs~\cite{nane2016survey}.
%\todo{As datasets continue to grow and compute power fails to have commensurate scaling,
%it is becoming obvious that dataflow architectures will provide one avenue for domain experts to
%squeeze out better performance for their algorithms.}

\begin{figure}
\centering

\newsavebox{\gemm}
\begin{lrbox}{\gemm}
\begin{lstlisting}[language=Spatial,linewidth=0.88\columnwidth]
// Custom floating point format
// 11 mantissa, 5 exponent bits
type Half = FltPt[11,5]

def main(args: Array[String]) {

  // Load data from files
  val a: Matrix[Half] = loadMatrix[Half](args(0))
  val b: Matrix[Half] = loadMatrix[Half](args(1))

  // Allocate space on accelerator DRAM
  val A = DRAM[Half](a.rows,a.cols)
  val B = DRAM[Half](b.rows,b.cols)
  val C = DRAM[Half](a.rows,b.cols)

  // Create explicit design parameters
  val M = 128 (64, 1024)  // Tile size for output rows
  val N = 128 (64, 1024)  // Tile size for output cols
  val P = 128 (64, 1024)  // Tile size for common
  val PAR_K  = 2 (1, 8)   // Unroll factor of k
  val PAR_J  = 2 (1, 16)  // Unroll factor of j

  // Transfer data to accelerator DRAM
  sendMatrix(A, a)
  sendMatrix(B, b)

  // Specify the accelerator design
  Accel {
    // Produce C in M x N tiles
    Foreach(A.rows by M, B.cols by N){ (ii,jj) =>
      val tileC = SRAM[Half](M, N)

      // Combine intermediates across common dimension
      MemReduce(tileC)(A.cols by P){ kk =>
        // Allocate on-chip scratchpads
        val tileA = SRAM[Half](M, P)
        val tileB = SRAM[Half](P, N)
        val accum = SRAM[Half](M, N)

        // Load tiles of A and B from DRAM
        tileA load A(ii::ii+M, kk::kk+P)  // M x P
        tileB load B(kk::kk+P, jj::jj+N)  // P x N

        // Combine intermediates across a chunk of P
        MemReduce(accum)(P by 1 par PAR_K){ k =>
          val partC = SRAM[Half](M, N)
          Foreach(M by 1, N by 1 par PAR_J){ (i,j) =>
            partC(i,j) = tileA(i,k) * tileB(k,j)
          }
          partC
        // Combine intermediates with element-wise add
        }{(a,b) => a + b }
      }{(a,b) => a + b }

      // Store the tile of C to DRAM
      C(ii::ii+M, jj::jj+N) store tileC
    }
  }

  // Save the result to another file
  saveMatrix(args(2), getMatrix(C))
}
\end{lstlisting}
\end{lrbox}

\hspace{-15pt}\usebox{\gemm}
\caption{Basic parameterized matrix-matrix multiplication ($C$~=~$A~\cdot~B$) implemented in Spatial.}
\label{fig:matmult}
\end{figure}


This paper presents a practical framework for automatic generation of
efficient FPGA accelerators. Figure~\ref{fig:system-diag} describes
our overall system architecture. The input to our system is an
application described in a high-level language using high-level
parallel patterns like \emph{map}, \emph{reduce},
\emph{filter}, and \emph{groupBy}~\cite{optiml}.
%\christos{add ref to such the  language(s)}
Parallel patterns serve the dual purpose of raising the
level of abstraction for the
programmer~\cite{ecoop13sujeeth,pldi13halide}, and providing richer
semantic information to the compiler~\cite{delite-tecs14}. These
constructs are then automatically lowered into our hardware definition
language (HDL) that explicitly captures information on parallelism,
locality, and memory access pattern at all levels of nesting using
parameterizable architectural templates (Step 1 in
Figure~\ref{fig:system-diag}). Step 1 performs high-level
optimizations like loop fusion and tiling transformations. The output
of step 1 is a tiled representation of the input design expressed in
our HDL. Note that tiling here includes both loop and data tiling.

%Languages like Java and C++ were able
%to enhance software engineering by lifting the abstraction, and their respective compilers used their view
%of the entire program space to perform optimizations that are otherwise too convoluted for a person
%to do when handwriting assembly. 
In this work, we first summarize high-level language abstractions required to create a new high-level synthesis language from the ground up, including syntax for managing memory, control, and accelerator-host interfaces on a reconfigurable architecture.
We suggest that this ``clean slate'' approach to high-level synthesis language design leads to a language which is semantically cleaner when targeting reconfigurable architectures, particularly when optimizing for data locality and parallelism. 
These abstractions help programmer productivity and allow both the user and compiler to more easily optimize designs for improved performance. 


We then describe a new domain specific language (DSL) and compiler framework called Spatial which implements these abstractions to support higher level, performance oriented hardware accelerator design. 
Figure~\ref{fig:matmult} shows an example of a basic implementation of matrix multiplication in Spatial.
As this figure shows, Spatial code is like existing HLS languages in that programs are untimed and the language encourages accelerator designs to be expressed in terms of nested loops. However, unlike existing HLS tools, Spatial gives users more explicit control over the memory hierarchy through a library of on-chip and off-chip memory templates (e.g. the \texttt{DRAM} and \texttt{SRAM} in Figure~\ref{fig:matmult}).
Spatial automatically pipelines arbitrarily nested loops, and banks, buffers, and duplicates memories for the user based on parallel access patterns by default. 
This is in contrast to modern HLS tools, which largely rely on the user to add explicit pragmas to their code in order make these optimizations.
Spatial also supports tuning of parameterized designs via automated design space exploration (DSE).
Unlike prior approaches~\cite{dhdl} which use variance-prone heuristic random search, Spatial employs an active machine learning framework called HyperMapper \cite{Bodin2016:PACT16} to drive exploration.
This tuning allows a single accelerator design to be quickly ported across target architectures and vendors with ease.

When targeting FPGAs, Spatial generates optimized, synthesizable Chisel code along with C++ code which can be used on a host CPU to administrate initialization and execution of the accelerator on the target FPGA. 
Spatial currently supports Xilinx Ultrascale+ VU9P FPGAs on Amazon's EC2 F1 Instances, Xilinx Zynq-7000 and Ultrascale+ ZCU102 SoCs, and Altera DE1 and Arria 10 SoCs.
The constructs in Spatial are general across reconfigurable architectures, meaning Spatial programs can also be used to target CGRAs. In this paper, we demonstrate this by targeting our recently proposed Plasticine CGRA~\cite{plasticine}.






%Similarly, we introduce Spatial, a complete Domain Specific Language (DSL)
%that both provides the high-level abstractions needed to simplify programming spatial architectures and
%the hardware-specific optimizations over the entire application that are otherwise extremely challenging
%to do by hand in a low level language.

The contributions of this paper are as follows:
\vspace{-5pt}
\begin{itemize}
  \item We discuss the abstractions required to describe target-agnostic accelerator designs for reconfigurable architectures (Section~\ref{background}). We then describe Spatial's implementation of these constructs (Section~\ref{language}) and the optimizations that these abstraction enables in the Spatial compiler (Section~\ref{compiler}).

  \vspace{5pt}

  \item We describe an improved method of fast, automated design parameter space exploration using HyperMapper (Section~\ref{dse}). This approach is evaluated in Section~\ref{evaluation}.

  \vspace{5pt}

  \item We evaluate Spatial's ability to efficiently express a wide variety of applications and
    target multiple architectures from the same source code. We demonstrate Spatial targeting two FPGAs and the Plasticine CGRA. 
    We quantitatively compare Spatial to SDAccel on the VU9P FPGA on a diverse set of benchmarks (Section~\ref{evaluation}), showing a geometric mean speedup of $2.9\times$ with 42\% less code.
    We provide a qualitative comparison of Spatial to other related work in Section~\ref{related}.
\end{itemize}